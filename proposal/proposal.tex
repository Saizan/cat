\documentclass{article}



\usepackage[utf8]{inputenc}

\usepackage{natbib}
\usepackage{hyperref}

\usepackage{graphicx}

\usepackage[colorinlistoftodos]{todonotes}

\usepackage{parskip}
\setlength{\parskip}{10pt}

\usepackage{tikz}
\usetikzlibrary{arrows, decorations.markings}

\usepackage{chngcntr}
\counterwithout{figure}{section}



\begin{document}


\begin{titlepage}


\centering


{\scshape\LARGE Master thesis project proposal\\}

\vspace{0.5cm}

{\huge\bfseries Title of the project\\}

\vspace{2cm}

{\Large name and email adress of student 1\\}

\vspace{0.2cm}

{\Large name and email adress of student 2\\}

\vspace{1.0cm}

{\large Suggested Supervisor at CSE (if you have one, otherwise skip this row): Name of supervisor\\}

\vspace{1.5cm}

{\large Supervisor at Company (if applicable): Name of supervisor, name of company\\}

\vspace{1.5cm}

{\large Relevant completed courses student 1:\par}

{\itshape List (course code, name of course)\\}

\vspace{1.5cm}

{\large Relevant completed courses student 2:\par}

{\itshape List (course code, name of course)\\}

\vfill



\vfill

{\large \today\\}


\end{titlepage}

\section{Introduction}


Briefly describe and motivate the project, and convince the reader of the importance of the proposed thesis work.
A good introduction will answer these questions: Why is addressing these challenges significant for gaining new knowledge
in the studied domain? How and where can this new knowledge be applied?



\section{Problem}



This section is optional. It may be used if there is a need to describe the problem that you want to solve in more technical
detail and if this problem description is too extensive to fit in the introduction.




\section{Context}



Use one or two relevant and high quality references for providing evidence from the literature that the proposed study indeed
includes scientific and engineering challenges, or is related to existing ones. Convince the reader that the problem addressed
in this thesis has not been solved prior to this project.





\section{Goals and Challenges}



Describe your contribution with respect to concepts, theory and technical goals. Ensure that the scientific and engineering
challenges stand out so that the reader can easily recognize that you are planning to solve an advanced problem.


\section{Approach}



Various scientific approaches are appropriate for different challenges and project goals. Outline and justify the ones that
you have selected. For example, when your project considers systematic data collection, you need to explain how you will analyze
the data, in order to address your challenges and project goals.

One scientific approach is to use formal models and rigorous
mathematical argumentation to address aspects like correctness and efficiency. If this is relevant, describe the related algorithmic
subjects, and how you plan to address the studied problem. For example, if your plan is to study the problem from a computability aspect,
address the relevant issues, such as algorithm and data structure design, complexity analysis, etc.  If you plan to develop and
evaluate a prototype, briefly describe your plans to design, implement, and evaluate your prototype by reviewing at most two relevant
issues, such as key functionalities and their evaluation criteria.

The design and implementation should specify prototype properties,
such as functionalities and performance goals, e.g., scalability, memory, energy. Motivate key design selection, with respect to state
of the art and existing platforms, libraries, etc.

When discussing evaluation criteria, describe the testing environment, e.g., test-bed
experiments, simulation, and user studies, which you plan to use when assessing your prototype. Specify key tools, and preliminary
test-case scenarios. Explain how and why you plan to use the evaluation criteria in order to demonstrate the functionalities and
design goals. Explain how you plan to compare your prototype to the state of the art using the proposed test-case evaluation scenarios
and benchmarks.



\section{References}



%\bibliographystyle{plain}

%\bibliography{references}



Reference all sources that are cited in your proposal using, e.g. the APA, Harvard2, or IEEE3 style.



\end{document}
