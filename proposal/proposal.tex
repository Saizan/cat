\documentclass{article}



\usepackage[utf8]{inputenc}

\usepackage{natbib}
\usepackage[hidelinks]{hyperref}

\usepackage{graphicx}

\usepackage{parskip}
\usepackage{multicol}
\usepackage{amsmath,amssymb}
\usepackage[toc,page]{appendix}
\usepackage{xspace}

% \setlength{\parskip}{10pt}

% \usepackage{tikz}
% \usetikzlibrary{arrows, decorations.markings}

% \usepackage{chngcntr}
% \counterwithout{figure}{section}

\usepackage{chalmerstitle}
\newcommand{\subsubsubsection}[1]{\textbf{#1}}
\newcommand{\WIP}{\textbf{WIP}}

\newcommand{\coloneqq}{\mathrel{\vcenter{\baselineskip0.5ex \lineskiplimit0pt
                     \hbox{\scriptsize.}\hbox{\scriptsize.}}}%
  =}

\newcommand{\defeq}{\triangleq}
%% Alternatively:
%% \newcommand{\defeq}{≔}
\newcommand{\bN}{\mathbb{N}}
\newcommand{\bC}{\mathbb{C}}
\newcommand{\bX}{\mathbb{X}}
% \newcommand{\to}{\rightarrow}
\newcommand{\mto}{\mapsto}
\newcommand{\UU}{\ensuremath{\mathcal{U}}\xspace}
\let\type\UU
\newcommand{\MCU}{\UU}
\newcommand{\nomen}[1]{\emph{#1}}
\newcommand{\todo}[1]{\textit{#1}}
\newcommand{\comp}{\circ}
\newcommand{\x}{\times}
\newcommand\inv[1]{#1\raisebox{1.15ex}{$\scriptscriptstyle-\!1$}}
\newcommand{\tp}{\;\mathord{:}\;}
\newcommand{\Type}{\mathcal{U}}

\usepackage{graphicx}
\makeatletter
\newcommand{\shorteq}{%
  \settowidth{\@tempdima}{-}% Width of hyphen
  \resizebox{\@tempdima}{\height}{=}%
}
\makeatother
\newcommand{\var}[1]{\ensuremath{\mathit{#1}}}
\newcommand{\Hom}{\var{Hom}}
\newcommand{\fmap}{\var{fmap}}
\newcommand{\bind}{\var{bind}}
\newcommand{\join}{\var{join}}
\newcommand{\omap}{\var{omap}}
\newcommand{\pure}{\var{pure}}
\newcommand{\idFun}{\var{id}}
\newcommand{\Sets}{\var{Sets}}
\newcommand{\Set}{\var{Set}}
\newcommand{\hSet}{\var{hSet}}
\newcommand{\id}{\var{id}}
\newcommand{\isEquiv}{\var{isEquiv}}
\newcommand{\idToIso}{\var{idToIso}}
\newcommand{\isSet}{\var{isSet}}
\newcommand{\isContr}{\var{isContr}}
\newcommand\Object{\var{Object}}
\newcommand\Functor{\var{Functor}}
\newcommand\isProp{\var{isProp}}
\newcommand\propPi{\var{propPi}}
\newcommand\propSig{\var{propSig}}
\newcommand\PreCategory{\var{PreCategory}}
\newcommand\IsPreCategory{\var{IsPreCategory}}
\newcommand\isIdentity{\var{isIdentity}}
\newcommand\propIsIdentity{\var{propIsIdentity}}
\newcommand\IsCategory{\var{IsCategory}}
\newcommand\Gl{\var{\lambda}}
\newcommand\lemPropF{\var{lemPropF}}
\newcommand\isPreCategory{\var{isPreCategory}}
\newcommand\congruence{\var{cong}}
\newcommand\identity{\var{identity}}
\newcommand\isequiv{\var{isequiv}}
\newcommand\qinv{\var{qinv}}
\newcommand\fiber{\var{fiber}}
\newcommand\shuffle{\var{shuffle}}
\newcommand\Univalent{\var{Univalent}}
\newcommand\refl{\var{refl}}
\newcommand\isoToId{\var{isoToId}}
\newcommand\rrr{\ggg}
\newcommand\fish{\mathrel{\wideoverbar{\rrr}}}
\newcommand\fst{\var{fst}}
\newcommand\snd{\var{snd}}
\newcommand\Path{\var{Path}}
\newcommand\Category{\var{Category}}
\newcommand\TODO[1]{TODO: \emph{#1}}
\newcommand*{\QED}{\hfill\ensuremath{\square}}%
\newcommand\uexists{\exists!}
\newcommand\Arrow{\var{Arrow}}
\newcommand\NTsym{\var{NT}}
\newcommand\NT[2]{\NTsym\ #1\ #2}
\newcommand\Endo[1]{\var{Endo}\ #1}
\newcommand\EndoR{\mathcal{R}}


\title{Category Theory and Cubical Type Theory}
\author{Frederik Hanghøj Iversen}
\authoremail{hanghj@student.chalmers.se}
\supervisor{Thierry Coquand}
\supervisoremail{coquand@chalmers.se}
\cosupervisor{Andrea Vezzosi}
\cosupervisoremail{vezzosi@chalmers.se}
\institution{Chalmers University of Technology}

\begin{document}

\maketitle
%
\section{Introduction}
%
Functional extensionality and univalence is not expressible in
\nomen{Intensional Martin Löf Type Theory} (ITT). This poses a severe limitation
on both 1) what is \emph{provable} and 2) the \emph{reusability} of proofs.
Recent developments have, however, resulted in \nomen{Cubical Type Theory} (CTT)
which permits a constructive proof of these two important notions.

Furthermore an extension has been implemented for the proof assistant Agda
(\cite{agda}) that allows us to work in such a ``cubical setting''. This project
will be concerned with exploring the usefulness of this extension. As a
case-study I will consider \nomen{category theory}. This will serve a dual
purpose: First off category theory is a field where the notion of functional
extensionality and univalence wil be particularly useful. Secondly, Category
Theory gives rise to a \nomen{model} for CTT.

The project will consist of two parts: The first part will be concerned with
formalizing concepts from category theory. The focus will be on formalizing
parts that will be useful in the second part of the project: Showing that
\nomen{Cubical Sets} give rise to a model of CTT.
%
\section{Problem}
%
In the following two subsections I present two examples that illustrate the
limitation inherent in ITT and by extension to the expressiveness of Agda.
%
\subsection{Functional extensionality}
Consider the functions:
%
\begin{multicols}{2}
$f \defeq (n : \bN) \mapsto (0 + n : \bN)$

$g \defeq (n : \bN) \mapsto (n + 0 : \bN)$
\end{multicols}
%
$n + 0$ is definitionally equal to $n$. We call this \nomen{definitional
equality} and write $n + 0 = n$ to assert this fact. We call it definitional
equality because the \emph{equality} arises from the \emph{definition} of $+$
which is:
%
\newcommand{\suc}[1]{\mathit{suc}\ #1}
\begin{align*}
  +           & : \bN \to \bN              \\
  n + 0       & \defeq n                   \\
  n + (\suc{m}) & \defeq \suc{(n + m)}
\end{align*}
%
Note that $0 + n$ is \emph{not} definitionally equal to $n$. $0 + n$ is in
normal form. I.e.; there is no rule for $+$ whose left-hand-side matches this
expression. We \emph{do}, however, have that they are \nomen{propositionally}
equal. We write $n + 0 \equiv n$ to assert this fact. Propositional equality
means that there is a proof that exhibits this relation. Since equality is a
transitive relation we have that $n + 0 \equiv 0 + n$.

Unfortunately we don't have $f \equiv g$.\footnote{Actually showing this is
outside the scope of this text. Essentially it would involve giving a model
for our type theory that validates all our axioms but where $f \equiv g$ is
not true.} There is no way to construct a proof asserting the obvious
equivalence of $f$ and $g$ -- even though we can prove them equal for all
points. This is exactly the notion of equality of functions that we are
interested in; that they are equal for all inputs. We call this
\nomen{pointwise equality}, where the \emph{points} of a function refers
to it's arguments.

In the context of category theory the principle of functional extensionality is
for instance useful in the context of showing that representable functors are
indeed functors. The representable functor for a category $\bC$ and a fixed
object in $A \in \bC$ is defined to be:
%
\begin{align*}
\fmap \defeq X \mapsto \Hom_{\bC}(A, X)
\end{align*}
%
The proof obligation that this satisfies the identity law of functors
($\fmap\ \idFun \equiv \idFun$) becomes:
%
\begin{align*}
\Hom(A, \idFun_{\bX}) = (g \mapsto \idFun \comp g) \equiv \idFun_{\Sets}
\end{align*}
%
One needs functional extensionality to ``go under'' the function arrow and apply
the (left) identity law of the underlying category to proove $\idFun \comp g
\equiv g$ and thus closing the above proof.
%
\iffalse
I also want to talk about:
\begin{itemize}
\item
  Foundational systems
\item
  Theory vs. metatheory
\item
  Internal type theory
\end{itemize}
\fi
\subsection{Equality of isomorphic types}
%
Let $\top$ denote the unit type -- a type with a single constructor. In the
propositions-as-types interpretation of type theory $\top$ is the proposition
that is always true. The type $A \x \top$ and $A$ has an element for each $a :
A$. So in a sense they are the same. The second element of the pair does not add
any ``interesting information''. It can be useful to identify such types. In
fact, it is quite commonplace in mathematics. Say we look at a set $\{x \mid
\phi\ x \land \psi\ x\}$ and somehow conclude that $\psi\ x \equiv \top$ for all
$x$. A mathematician would immediately conclude $\{x \mid \phi\ x \land
\psi\ x\} \equiv \{x \mid \phi\ x\}$ without thinking twice. Unfortunately such
an identification can not be performed in ITT.

More specifically; what we are interested in is a way of identifying types that
are in a one-to-one correspondence. We say that such types are
\nomen{isomorphic} and write $A \cong B$ to assert this.

To prove two types isomorphic is to give an \nomen{isomorphism} between them.
That is, a function $f : A \to B$ with an inverse $f^{-1} : B \to A$, i.e.:
$f^{-1} \comp f \equiv id_A$. If such a function exist we say that $A$ and $B$
are isomorphic and write $A \cong B$.

Furthermore we want to \emph{identify} such isomorphic types. This, we get from
the principle of univalence:\footnote{It's often referred to as the univalence
axiom, but since it is not an axiom in this setting but rather a theorem I
refer to this just as a `principle'.}
%
$$(A \cong B) \cong (A \equiv B)$$
%
\subsection{Formalizing Category Theory}
%
The above examples serve to illustrate the limitation of Agda. One case where
these limitations are particularly prohibitive is in the study of Category
Theory. At a glance category theory can be described as ``the mathematical study
of (abstract) algebras of functions'' (\cite{awodey-2006}). So by that token
functional extensionality is particularly useful for formulating Category
Theory. In Category theory it is also common to identify isomorphic structures
and this is exactly what we get from univalence.

\subsection{Cubical model for Cubical Type Theory}
%
A model is a way of giving meaning to a formal system in a \emph{meta-theory}. A
typical example of a model is that of sets as models for predicate logic. Thus
set-theory becomes the meta-theory of the formal language of predicate logic.

In the context of a given type theory and restricting ourselves to
\emph{categorical} models a model will consist of mapping `things' from the
type-theory (types, terms, contexts, context morphisms) to `things' in the
meta-theory (objects, morphisms) in such a way that the axioms of the
type-theory (typing-rules) are validated in the meta-theory. In
\cite{dybjer-1995} the author describes a way of constructing such models for
dependent type theory called \emph{Categories with Families} (CwFs).

In \cite{bezem-2014} the authors devise a CwF for Cubical Type Theory. This
project will study and formalize this model. Note that I will \emph{not} aim to
formalize CTT itself and therefore also not give the formal translation between
the type theory and the meta-theory. Instead the translation will be accounted
for informally.

The project will formalize CwF's. It will also define what pieces of data are
needed for a model of CTT (without explicitly showing that it does in fact model
CTT). It will then show that a CwF gives rise to such a model. Furthermore I
will show that cubical sets are presheaf categories and that any presheaf
category is itself a CwF. This is the precise way by which the project aims to
provide a model of CTT. Note that this formalization specifcally does not
mention the language of CTT itself. Only be referencing this previous work do we
arrive at a model of CTT.
%
\section{Context}
%
In \cite{bezem-2014} a categorical model for cubical type theory is presented.
In \cite{cohen-2016} a type-theory where univalence is expressible is presented.
The categorical model in the previous reference serve as a model of this type
theory. So these two ideas are closely related. Cubical type theory arose out of
\nomen{Homotopy Type Theory} (\cite{hott-2013}) and is also of interest as a
foundation of mathematics (\cite{voevodsky-2011}).

An implementation of cubical type theory can be found as an extension to Agda.
This is due to \citeauthor{cubical-agda}. This, of course, will be central to
this thesis.

The idea of formalizing Category Theory in proof assistants is not a new
idea\footnote{There are a multitude of these available online. Just as first
reference see this question on Math Overflow: \cite{mo-formalizations}}. The
contribution of this thesis is to explore how working in a cubical setting will
make it possible to prove more things and to reuse proofs.

There are alternative approaches to working in a cubical setting where one can
still have univalence and functional extensionality. One option is to postulate
these as axioms. This approach, however, has other shortcomings, e.g.; you lose
\nomen{canonicity} (\cite{huber-2016}). Canonicity means that any well-type
term will (under evaluation) reduce to a \emph{canonical} form. For example for
an integer $e : \bN$ it will be the case that $e$ is definitionally equal to $n$
applications of $\mathit{suc}$ to $0$ for some $n$; $e = \mathit{suc}^n\ 0$.
Without canonicity terms in the language can get ``stuck'' when they are
evaluated.

Another approach is to use the \emph{setoid interpretation} of type theory
(\cite{hofmann-1995,huber-2016}). Types should additionally `carry around' an
equivalence relation that should serve as propositional equality. This approach
has other drawbacks; it does not satisfy all judgemental equalites of type
theory and is cumbersome to work with in practice (\cite[p. 4]{huber-2016}).
%
\section{Goals and Challenges}
%
In summary, the aim of the project is to:
%
\begin{itemize}
\item
Formalize Category Theory in Cubical Agda
\item
Formalize Cubical Sets in Agda
% \item
% Formalize Cubical Type Theory in Agda
\item
Show that Cubical Sets are a model for Cubical Type Theory
\end{itemize}
%
The formalization of category theory will focus on extracting the elements from
Category Theory that we need in the latter part of the project. In doing so I'll
be gaining experience with working with Cubical Agda. Equality proofs using
cubical Agda can be tricky, so working with that will be a challenge in itself.
Most of the proofs in the context of cubical models I will formalize are based
on previous work. Those proofs, however, are not formalized in a proof
assistant.

One particular challenge in this context is that in a cubical setting there can
be multiple distinct terms that inhabit a given equality proof.\footnote{This is
in contrast with ITT where one \emph{can} have \nomen{Uniqueness of identity proofs}
(\cite[p. 4]{huber-2016}).} This means that the choice for a given equality
proof can influence later proofs that refer back to said proof. This is new and
relatively unexplored territory.

Another challenge is that Category Theory is something that I only know the
basics of. So learning the necessary concepts from Category Theory will also be
a goal and a challenge in itself.

After this has been implemented it would also be possible to formalize Cubical
Type Theory and formally show that Cubical Sets are a model of this. I do not
intend to formally implement the language of dependent type theory in this
project.

The thesis shall conclude with a discussion about the benefits of Cubical Agda.
%
\bibliographystyle{plainnat}
\nocite{cubical-demo}
\nocite{coquand-2013}
\bibliography{refs}
\begin{appendices}
\chapter{Planning report}
%
I have already implemented multiple essential building blocks for a
formalization of core-category theory. These concepts include:
%
\begin{itemize}
\item
Categories
\item
Functors
\item
Products
\item
Exponentials
\item
Natural transformations
\item
Concrete Categories
\subitem
Sets
\subitem
Cat
\subitem
Functor
\end{itemize}
%
Will all these things already in place it's my assessment that I am ahead of
schedule at this point.\footnote{I have omitted a lot of other things that
  follow easily from the above, e.g. a cartesian-closed category is simply one
  that has all products and exponentials.}

Here is a plan for my thesis work organized on a week-by-week basis.
%
\begin{center}
\centering
\begin{tabular}{@{}lll@{}}
Goal                           & Deadline    & Risk  1-5        \\ \hline
Yoneda embedding               & Feb 2nd     & 3  \\
Categories with families       & Feb 9th     & 4  \\
Presheafs $\Rightarrow$ CwF's  & Feb 16th    & 2  \\
Cubical Category               & Feb 23rd    & 3  \\
Writing seminar                & Mar 2nd     &    \\
Kan condition                  & Mar 9th     & 4  \\
Thesis outline and backlog     & Mar 16th    & 2  \\
Half-time report               & Mar 23rd    & 2  \\
                               & Mar 30th    &    \\
                               & Apr 6th     &    \\
                               & Apr 13th    &    \\
                               & Apr  20th   &    \\
Thesis draft                   & Apr 27th    & 2  \\
Writing seminar 2              & May 4th     &    \\
Presentation                   & May 11th    &    \\
Attend 1st presentation        & May 18th    &    \\
Attend 2nd presentation        & May 25th    &    \\
\end{tabular}
\end{center}
%
The first half part of my thesis-work will be focused on implementing core
elements of my Agda implementation. These core elements have been highlighted in
the above table. The elements noted there are the essential bits and pieces I
need to reach the ambitious goal of getting an implementation of a categorical
model for Cubical Type Theory. Along the way I will also have formalized
additional elements of more ``pure'' category theory. I will thus reach my goal
of formalizing (parts of) category theory.

The high risk-factors for CwF's and the Kan-condition is due to this being
somewhat uncharted territory for me at this point.

It's my plan that I will have formalized the core concepts needed around the
deadline for the half-time report which is due by March 23rd. Around this point
I will have a clearer idea of what additional things I need for a model of
category theory.

\chapter{Halftime report}
I've written this as an appendix because 1) the aim of the thesis changed
drastically from the planning report/proposal 2) partly I'm not sure how to
structure my thesis.

My work so far has very much focused on the formalization, i.e.\ coding. It's
unclear to me at this point what I should have in the final report. Here I will
describe what I have managed to formalize so far and what outstanding challenges
I'm facing.

\section{Implementation overview}
The overall structure of my project is as follows:

\begin{itemize}
\item Core categorical concepts
\subitem Categories
\subitem Functors
\subitem Products
\subitem Exponentials
\subitem Cartesian closed categories
\subitem Natural transformations
\subitem Yoneda embedding
\subitem Monads
\subsubitem Monoidal monads
\subsubitem Kleisli monads
\subsubitem Voevodsky's construction
\item Category of \ldots
\subitem Homotopy sets
\subitem Categories
\subitem Relations
\subitem Functors
\subitem Free category
\end{itemize}

I also started work on the category with families as well as the cubical
category as per the original goal of the thesis. However I have not gotten so
far with this.

In the following I will give an overview of overall results in each of these
categories (no pun).

As an overall design-guideline I've defined concepts in a such a way that the
``data'' and the ``laws'' about that data is split up in seperate modules. As an
example a category is defined to have two members: `raw` which is a collection
of the data and `isCategory` which asserts some laws about that data.

This allows me to reason about things in a more mathematical way, where one can
reason about two categories by simply focusing on the data. This is acheived by
creating a function embodying the ``equality principle'' for a given record. In
the case of monads, to prove two categories propositionally equal it enough to
provide a proof that their data is equal.

\subsection{Categories}
Defines the basic notion of a category. This definition closely follows that of
[HoTT]: That is, the standard definition of a category (data; objects, arrows,
composition and identity, laws; preservation of identity and composition) plus
the extra condition that it is univalent - namely that you can get an equality
of two objects from an isomorphism.

I make no distinction between a pre category and a real category (as in the
[HoTT]-sense). A pre category in my implementation would be a category sans the
witness to univalence.

I also prove that being a category is a proposition. This gives rise to an
equality principle for monads that focuses on the data-part.

I also show that the opposite category is indeed a category. (\WIP{} I have not
shown that univalence holds for such a construction)

I also show that taking the opposite is an involution.

\subsection{Functors}
Defines the notion of a functor - also split up into data and laws.

Propositionality for being a functor.

Composition of functors and the identity functor.

\subsection{Products}
Definition of what it means for an object to be a product in a given category.

Definition of what it means for a category to have all products.

\WIP{} Prove propositionality for being a product and having products.

\subsection{Exponentials}
Definition of what it means to be an exponential object.

Definition of what it means for a category to have all exponential objects.

\subsection{Cartesian closed categories}
Definition of what it means for a category to be cartesian closed; namely that
it has all products and all exponentials.

\subsection{Natural transformations}
Definition of transformations\footnote{Maybe this is a name I made up for a
  family of morphisms} and the naturality condition for these.

Proof that naturality is a mere proposition and the accompanying equality
principle. Proof that natural transformations are homotopic sets.

The identity natural transformation.

\subsection{Yoneda embedding}

The yoneda embedding is typically presented in terms of the category of
categories (cf. Awodey) \emph however this is not stricly needed - all we need
is what would be the exponential object in that category - this happens to be
functors and so this is how we define the yoneda embedding.

\subsection{Monads}

Defines an equivalence between these two formulations of a monad:

\subsubsection{Monoidal monads}

Defines the standard monoidal representation of a monad:

An endofunctor with two natural transformations (called ``pure'' and ``join'')
and some laws about these natural transformations.

Propositionality proofs and equality principle is provided.

\subsubsection{Kleisli monads}

A presentation of monads perhaps more familiar to a functional programer:

A map on objects and two maps on morphisms (called ``pure'' and ``bind'') and
some laws about these maps.

Propositionality proofs and equality principle is provided.

\subsubsection{Voevodsky's construction}

Provides construction 2.3 as presented in an unpublished paper by Vladimir
Voevodsky. This construction is similiar to the equivalence provided for the two
preceding formulations
\footnote{ TODO: I would like to include in the thesis some motivation for why
  this construction is particularly interesting.}

\subsection{Homotopy sets}
The typical category of sets where the objects are modelled by an Agda set
(henceforth ``$\Type$'') at a given level is not a valid category in this cubical
settings, we need to restrict the types to be those that are homotopy sets. Thus
the objects of this category are:
%
$$\hSet_\ell \defeq \sum_{A \tp \MCU_\ell} \isSet\ A$$
%
The definition of univalence for categories I have defined is:
%
$$\isEquiv\ (\hA \equiv \hB)\ (\hA \cong \hB)\ \idToIso$$
%
Where $\hA and \hB$ denote objects in the category. Note that this is stronger
than
%
$$(\hA \equiv \hB) \simeq (\hA \cong \hB)$$
%
Because we require that the equivalence is constructed from the witness to:
%
$$\id \comp f \equiv f \x f \comp \id \equiv f$$
%
And indeed univalence does not follow immediately from univalence for types:
%
$$(A \equiv B) \simeq (A \simeq B)$$
%
Because $A\ B \tp \Type$ whereas $\hA\ \hB \tp \hSet$.

For this reason I have shown that this category satisfies the following
equivalent formulation of being univalent:
%
$$\prod_{A \tp hSet} \isContr \left( \sum_{X \tp hSet} A \cong X \right)$$
%
But I have not shown that it is indeed equivalent to my former definition.
\subsection{Categories}
Note that this category does in fact not exist. In stead I provide the
definition of the ``raw'' category as well as some of the laws.

Furthermore I provide some helpful lemmas about this raw category. For instance
I have shown what would be the exponential object in such a category.

These lemmas can be used to provide the actual exponential object in a context
where we have a witness to this being a category. This is useful if this library
is later extended to talk about higher categories.

\subsection{Functors}
The category of functors and natural transformations. An immediate corrolary is
the set of presheaf categories.

\WIP{} I have not shown that the category of functors is univalent.

\subsection{Relations}
The category of relations. \WIP{} I have not shown that this category is
univalent. Not sure I intend to do so either.

\subsection{Free category}
The free category of a category. \WIP{} I have not shown that this category is
univalent.

\section{Current Challenges}
Besides the items marked \WIP{} above I still feel a bit unsure about what to
include in my report. Most of my work so far has been specifically about
developing this library. Some ideas:
%
\begin{itemize}
\item
  Modularity properties
\item
  Compare with setoid-approach to solve similiar problems.
\item
  How to structure an implementation to best deal with types that have no
  structure (propositions) and those that do (sets and everything above)
\end{itemize}
%
\section{Ideas for future developments}
\subsection{Higher categories}
I only have a notion of (1-)categories. Perhaps it would be nice to also
formalize higher categories.

\subsection{Hierarchy of concepts related to monads}
In Haskell the type-class Monad sits in a hierarchy atop the notion of a functor
and applicative functors. There's probably a similiar notion in the
category-theoretic approach to developing this.

As I have already defined monads from these two perspectives, it would be
interesting to take this idea even further and actually show how monads are
related to applicative functors and functors. I'm not entirely sure how this
would look in Agda though.

\subsection{Use formulation on the standard library}
I also thought it would be interesting to use this library to show certain
properties about functors, applicative functors and monads used in the Agda
Standard library. So I went ahead and tried to show that agda's standard
library's notion of a functor (along with suitable laws) is equivalent to my
formulation (in the category of homotopic sets). I ran into two problems here,
however; the first one is that the standard library's notion of a functor is
indexed by the object map:
%
$$
\Functor \tp (\Type \to \Type) \to \Type
$$
%
Where $\Functor\ F$ has the member:
%
$$
\fmap \tp (A \to B) \to F A \to F B
$$
%
Whereas the object map in my definition is existentially quantified:
%
$$
\Functor \tp \Type
$$
%
And $\Functor$ has these members:
\begin{align*}
F     & \tp \Type \to \Type \\
\fmap & \tp (A \to B) \to F A \to F B\}
\end{align*}
%

\end{appendices}
\end{document}
