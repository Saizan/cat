\section{Halftime report}
I've written this as an appendix because 1) the aim of the thesis changed
drastically from the planning report/proposal 2) partly I'm not sure how to
structure my thesis.

My work so far has very much focused on the formalization, i.e. coding. It's
unclear to me at this point what I should have in the final report. Here I will
describe what I have managed to formalize so far and what outstanding challenges
I'm facing.

\subsection{Implementation overview}
The overall structure of my project is as follows:

\begin{itemize}
\item Core categorical concepts
\subitem Categories
\subitem Functors
\subitem Products
\subitem Exponentials
\subitem Cartesian closed categories
\subitem Natural transformations
\subitem Yoneda embedding
\subitem Monads
\subsubitem Monoidal monads
\subsubitem Kleisli monads
\subsubitem Voevodsky's construction
\item Category of \ldots
\subitem Homotopy sets
\subitem Categories
\subitem Relations
\subitem Functors
\subitem Free category
\end{itemize}

I also started work on the category with families as well as the cubical
category as per the original goal of the thesis. However I have not gotten so
far with this.

In the following I will give an overview of overall results in each of these
categories (no pun).

As an overall design-guideline I've defined concepts in a such a way that the
``data'' and the ``laws'' about that data is split up in seperate modules. As an
example a category is defined to have two members: `raw` which is a collection
of the data and `isCategory` which asserts some laws about that data.

This allows me to reason about things in a more mathematical way, where one can
reason about two categories by simply focusing on the data. This is acheived by
creating a function embodying the ``equality principle'' for a given record. In
the case of monads, to prove two categories propositionally equal it enough to
provide a proof that their data is equal.

\subsubsection{Categories}
Defines the basic notion of a category. This definition closely follows that of
[HoTT]: That is, the standard definition of a category (data; objects, arrows,
composition and identity, laws; preservation of identity and composition) plus
the extra condition that it is univalent - namely that you can get an equality
of two objects from an isomorphism.

I make no distinction between a pre-category and a real category (as in the
[HoTT]-sense). A pre-category in my implementation would be a category sans the
witness to univalence.

I also prove that being a category is a proposition. This gives rise to an
equality principle for monads that focuses on the data-part.

I also show that the opposite category is indeed a category. (\WIP{} I have not
shown that univalence holds for such a construction)

I also show that taking the opposite is an involution.

\subsubsection{Functors}
Defines the notion of a functor - also split up into data and laws.

Propositionality for being a functor.

Composition of functors and the identity functor.

\subsubsection{Products}
Definition of what it means for an object to be a product in a given category.

Definition of what it means for a category to have all products.

\WIP{} Prove propositionality for being a product and having products.

\subsubsection{Exponentials}
Definition of what it means to be an exponential object.

Definition of what it means for a category to have all exponential objects.

\subsubsection{Cartesian closed categories}
Definition of what it means for a category to be cartesian closed; namely that
it has all products and all exponentials.

\subsubsection{Natural transformations}
Definition of transformations\footnote{Maybe this is a name I made up for a
  family of morphisms} and the naturality condition for these.

Proof that naturality is a mere proposition and the accompanying equality
principle. Proof that natural transformations are homotopic sets.

The identity natural transformation.

\subsubsection{Yoneda embedding}

The yoneda embedding is typically presented in terms of the category of
categories (cf. Awodey) \emph however this is not stricly needed - all we need
is what would be the exponential object in that category - this happens to be
functors and so this is how we define the yoneda embedding.

\subsubsection{Monads}

Defines an equivalence between these two formulations of a monad:

\subsubsubsection{Monoidal monads}

Defines the standard monoidal representation of a monad:

An endofunctor with two natural transformations (called ``pure'' and ``join'')
and some laws about these natural transformations.

Propositionality proofs and equality principle is provided.

\subsubsubsection{Kleisli monads}

A presentation of monads perhaps more familiar to a functional programer:

A map on objects and two maps on morphisms (called ``pure'' and ``bind'') and
some laws about these maps.

Propositionality proofs and equality principle is provided.

\subsubsubsection{Voevodsky's construction}

Provides construction 2.3 as presented in an unpublished paper by Vladimir
Voevodsky. This construction is similiar to the equivalence provided for the two
preceding formulations
\footnote{ TODO: I would like to include in the thesis some motivation for why
  this construction is particularly interesting.}

\subsubsection{Homotopy sets}
The typical category of sets where the objects are modelled by an Agda set
(henceforth ``$\Type$'') at a given level is not a valid category in this cubical
settings, we need to restrict the types to be those that are homotopy sets. Thus
the objects of this category are:
%
$$\hSet_\ell \defeq \sum_{A \tp \MCU_\ell} \isSet\ A$$
%
The definition of univalence for categories I have defined is:
%
$$\isEquiv\ (\hA \equiv \hB)\ (\hA \cong \hB)\ \idToIso$$
%
Where $\hA and \hB$ denote objects in the category. Note that this is stronger
than
%
$$(\hA \equiv \hB) \simeq (\hA \cong \hB)$$
%
Because we require that the equivalence is constructed from the witness to:
%
$$\id \comp f \equiv f \x f \comp \id \equiv f$$
%
And indeed univalence does not follow immediately from univalence for types:
%
$$(A \equiv B) \simeq (A \simeq B)$$
%
Because $A\ B \tp \Type$ whereas $\hA\ \hB \tp \hSet$.

For this reason I have shown that this category satisfies the following
equivalent formulation of being univalent:
%
$$\prod_{A \tp hSet} \isContr \left( \sum_{X \tp hSet} A \cong X \right)$$
%
But I have not shown that it is indeed equivalent to my former definition.
\subsubsection{Categories}
Note that this category does in fact not exist. In stead I provide the
definition of the ``raw'' category as well as some of the laws.

Furthermore I provide some helpful lemmas about this raw category. For instance
I have shown what would be the exponential object in such a category.

These lemmas can be used to provide the actual exponential object in a context
where we have a witness to this being a category. This is useful if this library
is later extended to talk about higher categories.

\subsubsection{Functors}
The category of functors and natural transformations. An immediate corrolary is
the set of presheaf categories.

\WIP{} I have not shown that the category of functors is univalent.

\subsubsection{Relations}
The category of relations. \WIP{} I have not shown that this category is
univalent. Not sure I intend to do so either.

\subsubsection{Free category}
The free category of a category. \WIP{} I have not shown that this category is
univalent.

\subsection{Current Challenges}
Besides the items marked \WIP{} above I still feel a bit unsure about what to
include in my report. Most of my work so far has been specifically about
developing this library. Some ideas:
%
\begin{itemize}
\item
  Modularity properties
\item
  Compare with setoid-approach to solve similiar problems.
\item
  How to structure an implementation to best deal with types that have no
  structure (propositions) and those that do (sets and everything above)
\end{itemize}
%
\subsection{Ideas for future developments}
\subsubsection{Higher categories}
I only have a notion of (1-)categories. Perhaps it would be nice to also
formalize higher categories.

\subsubsection{Hierarchy of concepts related to monads}
In Haskell the type-class Monad sits in a hierarchy atop the notion of a functor
and applicative functors. There's probably a similiar notion in the
category-theoretic approach to developing this.

As I have already defined monads from these two perspectives, it would be
interesting to take this idea even further and actually show how monads are
related to applicative functors and functors. I'm not entirely sure how this
would look in Agda though.

\subsubsection{Use formulation on the standard library}
I also thought it would be interesting to use this library to show certain
properties about functors, applicative functors and monads used in the Agda
Standard library. So I went ahead and tried to show that agda's standard
library's notion of a functor (along with suitable laws) is equivalent to my
formulation (in the category of homotopic sets). I ran into two problems here,
however; the first one is that the standard library's notion of a functor is
indexed by the object map:
%
$$
\Functor   & \tp (\Type \to \Type) \to \Type
$$
%
Where $\Functor\ F$ has the member:
%
$$
\fmap \tp (A \to B) \to F A \to F B
$$
%
Whereas the object map in my definition is existentially quantified:
%
$$
\Functor   & \tp \Type
$$
%
And $\Functor$ has these members:
\begin{align*}
F     & \tp \Type \to \Type \\
\fmap & \tp (A \to B) \to F A \to F B\}
\end{align*}
%
