\section{Introduction}
%
Functional extensionality and univalence is not expressible in
\nomen{Intensional Martin Löf Type Theory} (ITT). This poses a severe limitation
on both 1) what is \emph{provable} and 2) the \emph{reusability} of proofs.
Recent developments have, however, resulted in \nomen{Cubical Type Theory} (CTT)
which permits a constructive proof of these two important notions.

Furthermore an extension has been implemented for the proof assistant Agda
(\cite{agda}) that allows us to work in such a ``cubical setting''. This project
will be concerned with exploring the usefulness of this extension. As a
case-study I will consider \nomen{category theory}. This will serve a dual
purpose: First off category theory is a field where the notion of functional
extensionality and univalence wil be particularly useful. Secondly, Category
Theory gives rise to a \nomen{model} for CTT.

The project will consist of two parts: The first part will be concerned with
formalizing concepts from category theory. The focus will be on formalizing
parts that will be useful in the second part of the project: Showing that
\nomen{Cubical Sets} give rise to a model of CTT.
%
\section{Problem}
%
In the following two subsections I present two examples that illustrate the
limitation inherent in ITT and by extension to the expressiveness of Agda.
%
\subsection{Functional extensionality}
Consider the functions:
%
\begin{multicols}{2}
$f \defeq (n : \bN) \mapsto (0 + n : \bN)$

$g \defeq (n : \bN) \mapsto (n + 0 : \bN)$
\end{multicols}
%
$n + 0$ is definitionally equal to $n$. We call this \nomen{definitional
equality} and write $n + 0 = n$ to assert this fact. We call it definitional
equality because the \emph{equality} arises from the \emph{definition} of $+$
which is:
%
\newcommand{\suc}[1]{\mathit{suc}\ #1}
\begin{align*}
  +           & : \bN \to \bN              \\
  n + 0       & \defeq n                   \\
  n + (\suc{m}) & \defeq \suc{(n + m)}
\end{align*}
%
Note that $0 + n$ is \emph{not} definitionally equal to $n$. $0 + n$ is in
normal form. I.e.; there is no rule for $+$ whose left-hand-side matches this
expression. We \emph{do}, however, have that they are \nomen{propositionally}
equal. We write $n + 0 \equiv n$ to assert this fact. Propositional equality
means that there is a proof that exhibits this relation. Since equality is a
transitive relation we have that $n + 0 \equiv 0 + n$.

Unfortunately we don't have $f \equiv g$.\footnote{Actually showing this is
outside the scope of this text. Essentially it would involve giving a model
for our type theory that validates all our axioms but where $f \equiv g$ is
not true.} There is no way to construct a proof asserting the obvious
equivalence of $f$ and $g$ -- even though we can prove them equal for all
points. This is exactly the notion of equality of functions that we are
interested in; that they are equal for all inputs. We call this
\nomen{pointwise equality}, where the \emph{points} of a function refers
to it's arguments.

In the context of category theory the principle of functional extensionality is
for instance useful in the context of showing that representable functors are
indeed functors. The representable functor for a category $\bC$ and a fixed
object in $A \in \bC$ is defined to be:
%
\begin{align*}
\fmap \defeq X \mapsto \Hom_{\bC}(A, X)
\end{align*}
%
The proof obligation that this satisfies the identity law of functors
($\fmap\ \idFun \equiv \idFun$) becomes:
%
\begin{align*}
\Hom(A, \idFun_{\bX}) = (g \mapsto \idFun \comp g) \equiv \idFun_{\Sets}
\end{align*}
%
One needs functional extensionality to ``go under'' the function arrow and apply
the (left) identity law of the underlying category to proove $\idFun \comp g
\equiv g$ and thus closing the above proof.
%
\iffalse
I also want to talk about:
\begin{itemize}
\item
  Foundational systems
\item
  Theory vs. metatheory
\item
  Internal type theory
\end{itemize}
\fi
\subsection{Equality of isomorphic types}
%
Let $\top$ denote the unit type -- a type with a single constructor. In the
propositions-as-types interpretation of type theory $\top$ is the proposition
that is always true. The type $A \x \top$ and $A$ has an element for each $a :
A$. So in a sense they are the same. The second element of the pair does not add
any ``interesting information''. It can be useful to identify such types. In
fact, it is quite commonplace in mathematics. Say we look at a set $\{x \mid
\phi\ x \land \psi\ x\}$ and somehow conclude that $\psi\ x \equiv \top$ for all
$x$. A mathematician would immediately conclude $\{x \mid \phi\ x \land
\psi\ x\} \equiv \{x \mid \phi\ x\}$ without thinking twice. Unfortunately such
an identification can not be performed in ITT.

More specifically; what we are interested in is a way of identifying types that
are in a one-to-one correspondence. We say that such types are
\nomen{isomorphic} and write $A \cong B$ to assert this.

To prove two types isomorphic is to give an \nomen{isomorphism} between them.
That is, a function $f : A \to B$ with an inverse $f^{-1} : B \to A$, i.e.:
$f^{-1} \comp f \equiv id_A$. If such a function exist we say that $A$ and $B$
are isomorphic and write $A \cong B$.

Furthermore we want to \emph{identify} such isomorphic types. This, we get from
the principle of univalence:\footnote{It's often referred to as the univalence
axiom, but since it is not an axiom in this setting but rather a theorem I
refer to this just as a `principle'.}
%
$$(A \cong B) \cong (A \equiv B)$$
%
\subsection{Formalizing Category Theory}
%
The above examples serve to illustrate the limitation of Agda. One case where
these limitations are particularly prohibitive is in the study of Category
Theory. At a glance category theory can be described as ``the mathematical study
of (abstract) algebras of functions'' (\cite{awodey-2006}). So by that token
functional extensionality is particularly useful for formulating Category
Theory. In Category theory it is also common to identify isomorphic structures
and this is exactly what we get from univalence.

\subsection{Cubical model for Cubical Type Theory}
%
A model is a way of giving meaning to a formal system in a \emph{meta-theory}. A
typical example of a model is that of sets as models for predicate logic. Thus
set-theory becomes the meta-theory of the formal language of predicate logic.

In the context of a given type theory and restricting ourselves to
\emph{categorical} models a model will consist of mapping `things' from the
type-theory (types, terms, contexts, context morphisms) to `things' in the
meta-theory (objects, morphisms) in such a way that the axioms of the
type-theory (typing-rules) are validated in the meta-theory. In
\cite{dybjer-1995} the author describes a way of constructing such models for
dependent type theory called \emph{Categories with Families} (CwFs).

In \cite{bezem-2014} the authors devise a CwF for Cubical Type Theory. This
project will study and formalize this model. Note that I will \emph{not} aim to
formalize CTT itself and therefore also not give the formal translation between
the type theory and the meta-theory. Instead the translation will be accounted
for informally.

The project will formalize CwF's. It will also define what pieces of data are
needed for a model of CTT (without explicitly showing that it does in fact model
CTT). It will then show that a CwF gives rise to such a model. Furthermore I
will show that cubical sets are presheaf categories and that any presheaf
category is itself a CwF. This is the precise way by which the project aims to
provide a model of CTT. Note that this formalization specifcally does not
mention the language of CTT itself. Only be referencing this previous work do we
arrive at a model of CTT.
%
\section{Context}
%
In \cite{bezem-2014} a categorical model for cubical type theory is presented.
In \cite{cohen-2016} a type-theory where univalence is expressible is presented.
The categorical model in the previous reference serve as a model of this type
theory. So these two ideas are closely related. Cubical type theory arose out of
\nomen{Homotopy Type Theory} (\cite{hott-2013}) and is also of interest as a
foundation of mathematics (\cite{voevodsky-2011}).

An implementation of cubical type theory can be found as an extension to Agda.
This is due to \citeauthor{cubical-agda}. This, of course, will be central to
this thesis.

The idea of formalizing Category Theory in proof assistants is not a new
idea\footnote{There are a multitude of these available online. Just as first
reference see this question on Math Overflow: \cite{mo-formalizations}}. The
contribution of this thesis is to explore how working in a cubical setting will
make it possible to prove more things and to reuse proofs.

There are alternative approaches to working in a cubical setting where one can
still have univalence and functional extensionality. One option is to postulate
these as axioms. This approach, however, has other shortcomings, e.g.; you lose
\nomen{canonicity} (\cite{huber-2016}). Canonicity means that any well-type
term will (under evaluation) reduce to a \emph{canonical} form. For example for
an integer $e : \bN$ it will be the case that $e$ is definitionally equal to $n$
applications of $\mathit{suc}$ to $0$ for some $n$; $e = \mathit{suc}^n\ 0$.
Without canonicity terms in the language can get ``stuck'' when they are
evaluated.

Another approach is to use the \emph{setoid interpretation} of type theory
(\cite{hofmann-1995,huber-2016}). Types should additionally `carry around' an
equivalence relation that should serve as propositional equality. This approach
has other drawbacks; it does not satisfy all judgemental equalites of type
theory and is cumbersome to work with in practice (\cite[p. 4]{huber-2016}).
%
\section{Goals and Challenges}
%
In summary, the aim of the project is to:
%
\begin{itemize}
\item
Formalize Category Theory in Cubical Agda
\item
Formalize Cubical Sets in Agda
% \item
% Formalize Cubical Type Theory in Agda
\item
Show that Cubical Sets are a model for Cubical Type Theory
\end{itemize}
%
The formalization of category theory will focus on extracting the elements from
Category Theory that we need in the latter part of the project. In doing so I'll
be gaining experience with working with Cubical Agda. Equality proofs using
cubical Agda can be tricky, so working with that will be a challenge in itself.
Most of the proofs in the context of cubical models I will formalize are based
on previous work. Those proofs, however, are not formalized in a proof
assistant.

One particular challenge in this context is that in a cubical setting there can
be multiple distinct terms that inhabit a given equality proof.\footnote{This is
in contrast with ITT where one \emph{can} have \nomen{Uniqueness of identity proofs}
(\cite[p. 4]{huber-2016}).} This means that the choice for a given equality
proof can influence later proofs that refer back to said proof. This is new and
relatively unexplored territory.

Another challenge is that Category Theory is something that I only know the
basics of. So learning the necessary concepts from Category Theory will also be
a goal and a challenge in itself.

After this has been implemented it would also be possible to formalize Cubical
Type Theory and formally show that Cubical Sets are a model of this. I do not
intend to formally implement the language of dependent type theory in this
project.

The thesis shall conclude with a discussion about the benefits of Cubical Agda.
