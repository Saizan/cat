\chapter{Perspectives}
\section{Discussion}
In the previous chapter the practical aspects of proving things in Cubical Agda
were highlighted. I also demonstrated the usefulness of separating ``laws'' from
``data''. One of the reasons for this is that dependencies within types can lead
to very complicated goals. One technique for alleviating this was to prove that
certain types are mere propositions.

\subsection{Computational properties}
Another aspect (\TODO{That I actually did not highlight very well in the
  previous chapter}) is the computational nature of paths. Say we have
formalized this common result about monads:

\TODO{Some equation\ldots}

By transporting this to the Kleisli formulation we get a result that we can use
to compute with. This is particularly useful because the Kleisli formulation
will be more familiar to programmers e.g. those coming from a background in
Haskell. Whereas the theory usually talks about monoidal monads.

\TODO{Mention that with postulates we cannot do this}

\subsection{Reusability of proofs}
The previous example also illustrate how univalence unifies two otherwise
disparate areas: The category-theoretic study of monads; and monads as in
functional programming. Univalence thus allows one to reuse proofs. You could
say that univalence gives the developer two proofs for the price of one.

The introduction (section \S\ref{sec:context}) mentioned an often
employed-technique for enabling extensional equalities is to use the
setoid-interpretation. Nowhere in this formalization has this been necessary,
$\Path$ has been used globally in the project as propositional equality. One
interesting place where this becomes apparent is in interfacing with the Agda
standard library. Multiple definitions in the Agda standard library have been
designed with the setoid-interpretation in mind. E.g. the notion of ``unique
existential'' is indexed by a relation that should play the role of
propositional equality. Likewise for equivalence relations, they are indexed,
not only by the actual equivalence relation, but also by another relation that
serve as propositional equality.
%% Unfortunately we cannot use the definition of equivalences found in the
%% standard library to do equational reasoning directly. The reason for this is
%% that the equivalence relation defined there must be a homogenous relation,
%% but paths are heterogeneous relations.

In the formalization at present a significant amount of energy has been put
towards proving things that would not have been needed in classical Agda. The
proofs that some given type is a proposition were provided as a strategy to
simplify some otherwise very complicated proofs (e.g.
\ref{eq:proof-prop-IsPreCategory} and \label{eq:productPath}). Often these
proofs would not be this complicated. If the J-rule holds definitionally the
proof-assistant can help simplify these goals considerably. The lack of the
J-rule has a significant impact on the complexity of these kinds of proofs.

\TODO{Universe levels.}

\section{Future work}
\subsection{Agda \texttt{Prop}}
Jesper Cockx' work extending the universe-level-laws for Agda and the
\texttt{Prop}-type.

\subsection{Compiling Cubical Agda}
\label{sec:compiling-cubical-agda}
Compilation of program written in Cubical Agda is currently not supported. One
issue here is that the backends does not provide an implementation for the
cubical primitives (such as the path-type). This means that even though the
path-type gives us a computational interpretation of functional extensionality,
univalence, transport, etc., we do not have a way of actually using this to
compile our programs that use these primitives. It would be interesting to see
practical applications of this. The path between monads that this library
exposes could provide one particularly interesting case-study.

\subsection{Higher inductive types}
This library has not explored the usefulness of higher inductive types in the
context of Category Theory.
