\chapter{Perspectives}
\section{Discussion}
In the previous chapter the practical aspects of proving things in
Cubical Agda were highlighted. I also demonstrated the usefulness of
separating ``laws'' from ``data''. One of the reasons for this is that
dependencies within types can lead to very complicated goals. One
technique for alleviating this was to prove that certain types are
mere propositions.

\subsection{Computational properties}
The new contribution of cubical Agda is that it has a constructive
proof of functional extensionality\index{functional extensionality}
and univalence\index{univalence}. This means that in particular that
the type checker can reduce terms defined with these theorems. So one
interesting result of this development is how much this influenced the
development. In particular having a functional extensionality that
``computes'' should simplify some proofs.

I have tested this theory by using a feature of Agda where one can
mark certain bindings as being \emph{abstract}. This means that the
type-checker will not try to reduce that term further when
type-checking is performed. I tried making univalence and functional
extensionality abstract. It turns out that the conversion behaviour of
univalence is not used anywhere. For functional extensionality there
are two places in the whole solution where the reduction behaviour is
used to simplify some proofs. This is in showing that the maps between
the two formulations of monads are inverses. See the notes in this
module:
%
\begin{center}
\sourcelink{Cat.Category.Monad.Voevodsky}
\end{center}
%
I've also put this in a source listing in \ref{app:abstract-funext}. I
will not reproduce it in full here as the type is quite involved. The
method used to find in what places the computational behaviour of
these proofs are needed has the caveat of only working for places that
directly or transitively uses these two proofs. Fortunately though the
code is structured in such a way that this should be the case.
Nonetheless it is quite surprising that this computational behaviours
is not used more widely in the formalization.

Barring this, however, the computational behaviour of paths can still
be useful. E.g. if a programmer want's to reuse functions that operate
on a monoidal monads to work with a monad in the Kleisli form that
this programmer has specified. To make this idea concrete, say we are
given some function $f \tp \Kleisli \to T$ having a path between $p
\tp \Monoidal \equiv \Kleisli$ induces a map $\coe\ p \tp \Monoidal
\to \Kleisli$. We can compose $f$ with this map to get $f \comp
\coe\ p \tp \Monoidal \to T$. Of course, since that map was
constructed with an isomorphism these maps already exist and could be
used directly. So this is arguably only interesting when one wants to
prove properties of such functions.

\subsection{Reusability of proofs}
The previous example illustrate how univalence unifies two otherwise
disparate areas: The category-theoretic study of monads; and monads as
in functional programming. Univalence thus allows one to reuse proofs.
You could say that univalence gives the developer two proofs for the
price of one. As an illustration of this I proved that monads are
groupoids. I initially proved this for the Kleisli
formulation\footnote{Actually doing this directly turned out to be
  tricky as well, so I defined an equivalent formulation which was not
  formulated with a record, but purely with $\sum$-types.}. Since the
two formulations are equal under univalence, substitution directly
gives us that this also holds for the monoidal formulation. This of
course generalizes to any family $P \tp 𝒰 → 𝒰$ where $P$ is inhabited
at either formulation (i.e.\ either $P\ \Monoidal$ or $P\ \Kleisli$
holds).

The introduction (section \S\ref{sec:context}) mentioned an often
employed-technique for enabling extensional equalities is to use the
setoid-interpretation. Nowhere in this formalization has this been
necessary, $\Path$ has been used globally in the project as
propositional equality. One interesting place where this becomes
apparent is in interfacing with the Agda standard library. Multiple
definitions in the Agda standard library have been designed with the
setoid-interpretation in mind. E.g. the notion of ``unique
existential'' is indexed by a relation that should play the role of
propositional equality. Likewise for equivalence relations, they are
indexed, not only by the actual equivalence relation, but also by
another relation that serve as propositional equality.
%% Unfortunately we cannot use the definition of equivalences found in
%% the standard library to do equational reasoning directly. The
%% reason for this is that the equivalence relation defined there must
%% be a homogenous relation, but paths are heterogeneous relations.

In the formalization at present a significant amount of energy has
been put towards proving things that would not have been needed in
classical Agda. The proofs that some given type is a proposition were
provided as a strategy to simplify some otherwise very complicated
proofs (e.g. \ref{eq:proof-prop-IsPreCategory}
and \label{eq:productPath}). Often these proofs would not be this
complicated. If the J-rule holds definitionally the proof-assistant
can help simplify these goals considerably. The lack of the J-rule has
a significant impact on the complexity of these kinds of proofs.

\TODO{Universe levels.}

\subsection{Motifs}
An oft-used technique in this development is using based path
induction to prove certain properties. One particular challenge that
arises when doing so is that Agda is not able to automatically infer
the family that one wants to do induction over. For instance in the
proof $\var{sym}\ (\var{sym}\ p) ≡ p$ from \ref{eq:sym-invol} the
family that we chose to do induction over was $D\ b'\ p' \defeq
\var{sym}\ (\var{sym}\ p') ≡ p'$. However, if one interactively tries
to give this hole, all the information that Agda can provide is that
one must provide an element of $𝒰$. Agda could be more helpful in this
context, perhaps even infer this family in some situations. In this
very simple example this is of course not a big problem, but there are
examples in the source code where this gets more involved.

\section{Future work}
\subsection{Agda \texttt{Prop}}
Jesper Cockx' work extending the universe-level-laws for Agda and the
\texttt{Prop}-type.
\TODO{Do I want to include this?}

\subsection{Compiling Cubical Agda}
\label{sec:compiling-cubical-agda}
Compilation of program written in Cubical Agda is currently not
supported. One issue here is that the backends does not provide an
implementation for the cubical primitives (such as the path-type).
This means that even though the path-type gives us a computational
interpretation of functional extensionality, univalence, transport,
etc., we do not have a way of actually using this to compile our
programs that use these primitives. It would be interesting to see
practical applications of this. The path between monads that this
library exposes could provide one particularly interesting case-study.

\subsection{Higher inductive types}
This library has not explored the usefulness of higher inductive types
in the context of Category Theory.

\subsection{Initiality conjecture}
A fellow student here at Chalmers, Andreas Källberg, is currently
working on proving the initiality conjecture\TODO{Citation}. He will
be using this library to do so.
