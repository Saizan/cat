Functional extensionality and univalence is not expressible in
\nomen{Intensional Martin Löf Type Theory} (ITT). This poses a severe limitation
on both 1) what is \emph{provable} and 2) the \emph{reusability} of proofs.
Recent developments have, however, resulted in \nomen{Cubical Type Theory} (CTT)
which permits a constructive proof of these two important notions.

Furthermore an extension has been implemented for the proof assistant Agda
(\cite{agda}, \cite{cubical-agda}) that allows us to work in such a ``cubical
setting''. This thesis will explore the usefulness of this extension in the
context of category theory.
%
\section{Motivating examples}
%
In the following two sections I present two examples that illustrate some
limitations inherent in ITT and -- by extension -- Agda.
%
\subsection{Functional extensionality}
Consider the functions:
%
\begin{multicols}{2}
$f \defeq (n : \bN) \mapsto (0 + n : \bN)$

$g \defeq (n : \bN) \mapsto (n + 0 : \bN)$
\end{multicols}
%
$n + 0$ is \nomen{definitionally} equal to $n$ which we write as $n + 0 = n$.
This is also called \nomen{judgmental} equality. We call it definitional
equality because the \emph{equality} arises from the \emph{definition} of $+$
which is:
%
\newcommand{\suc}[1]{\mathit{suc}\ #1}
\begin{align*}
  +           & : \bN \to \bN              \\
  n + 0       & \defeq n                   \\
  n + (\suc{m}) & \defeq \suc{(n + m)}
\end{align*}
%
Note that $0 + n$ is \emph{not} definitionally equal to $n$. $0 + n$ is in
normal form. I.e.; there is no rule for $+$ whose left-hand-side matches this
expression. We \emph{do}, however, have that they are \nomen{propositionally}
equal which we write as $n + 0 \equiv n$. Propositional equality means that
there is a proof that exhibits this relation. Since equality is a transitive
relation we have that $n + 0 \equiv 0 + n$.

Unfortunately we don't have $f \equiv g$.\footnote{Actually showing this is
outside the scope of this text. Essentially it would involve giving a model
for our type theory that validates all our axioms but where $f \equiv g$ is
not true.} There is no way to construct a proof asserting the obvious
equivalence of $f$ and $g$ -- even though we can prove them equal for all
points. This is exactly the notion of equality of functions that we are
interested in; that they are equal for all inputs. We call this
\nomen{pointwise equality}, where the \emph{points} of a function refers
to it's arguments.

In the context of category theory functional extensionality is e.g. needed to
show that representable functors are indeed functors. The representable functor
for a category $\bC$ and a fixed object in $A \in \bC$ is defined to be:
%
\begin{align*}
\fmap \defeq X \mapsto \Hom_{\bC}(A, X)
\end{align*}
%
The proof obligation that this satisfies the identity law of functors
($\fmap\ \idFun \equiv \idFun$) becomes:
%
\begin{align*}
\Hom(A, \idFun_{\bX}) = (g \mapsto \idFun \comp g) \equiv \idFun_{\Sets}
\end{align*}
%
One needs functional extensionality to ``go under'' the function arrow and apply
the (left) identity law of the underlying category to proove $\idFun \comp g
\equiv g$ and thus closing the.
%
\subsection{Equality of isomorphic types}
%
Let $\top$ denote the unit type -- a type with a single constructor. In the
propositions-as-types interpretation of type theory $\top$ is the proposition
that is always true. The type $A \x \top$ and $A$ has an element for each $a :
A$. So in a sense they are the same. The second element of the pair does not add
any ``interesting information''. It can be useful to identify such types. In
fact, it is quite commonplace in mathematics. Say we look at a set $\{x \mid
\phi\ x \land \psi\ x\}$ and somehow conclude that $\psi\ x \equiv \top$ for all
$x$. A mathematician would immediately conclude $\{x \mid \phi\ x \land
\psi\ x\} \equiv \{x \mid \phi\ x\}$ without thinking twice. Unfortunately such
an identification can not be performed in ITT.

More specifically; what we are interested in is a way of identifying
\nomen{equivalent} types. I will return to the definition of equivalence later,
but for now, it is sufficient to think of an equivalence as a one-to-one
correspondence. We write $A \simeq B$ to assert that $A$ and $B$ are equivalent
types. The principle of univalence says that:
%
$$\mathit{univalence} \tp (A \simeq B) \simeq (A \equiv B)$$
%
In particular this allows us to construct an equality from an equivalence $\mathit{ua} \tp
(A \simeq B) \to (A \equiv B)$ and vice-versa.
\section{Formalizing Category Theory}
%
The above examples serve to illustrate the limitation of Agda. One case where
these limitations are particularly prohibitive is in the study of Category
Theory. At a glance category theory can be described as ``the mathematical study
of (abstract) algebras of functions'' (\cite{awodey-2006}). So by that token
functional extensionality is particularly useful for formulating Category
Theory. In Category theory it is also common to identify isomorphic structures
and this is exactly what we get from univalence. In fact we can formulate this
requirement within our formulation of categories by requiring the
\emph{categories} themselves to be univalent as we shall see.

\section{Context}
%
\begin{verbatim}
Inspiration:
* Awodey - formulation of categories
* HoTT   - sketch of homotopy proofs
\end{verbatim}
The idea of formalizing Category Theory in proof assistants is not new. There
are a multitude of these available online. Just as first reference see this
question on Math Overflow: \cite{mo-formalizations}. Notably these two implementations of category theory in Agda:
\begin{itemize}
\item
\url{https://github.com/copumpkin/categories} - setoid interpretation
\item
\url{https://github.com/pcapriotti/agda-categories} - homotopic setting with postulates
\item
\url{https://github.com/pcapriotti/agda-categories} - homotopic setting in coq
\item
\url{https://github.com/mortberg/cubicaltt} - homotopic setting in \texttt{cubicaltt}
\end{itemize}
The contribution of this
thesis is to explore how working in a cubical setting will make it possible to
prove more things and to reuse proofs.

There are alternative approaches to working in a cubical setting where one can
still have univalence and functional extensionality. One option is to postulate
these as axioms. This approach, however, has other shortcomings, e.g.; you lose
\nomen{canonicity} (\cite{huber-2016}). Canonicity means that any well-typed
term evaluates to a \emph{canonical} form. For example for a closed term $e :
\bN$ it will be the case that $e$ reduces to $n$ applications of $\mathit{suc}$
to $0$ for some $n$; $e = \mathit{suc}^n\ 0$. Without canonicity terms in the
language can get ``stuck'' -- meaning that they do not reduce to a canonical
form.

Another approach is to use the \emph{setoid interpretation} of type theory
(\cite{hofmann-1995,huber-2016}). With this approach one works with
\nomen{extensionals sets} $(X, \sim)$, that is a type $X \tp \MCU$ and an
equivalence relation $\sim$.

Types should additionally `carry around' an equivalence relation that serve as
propositional equality. This approach has other drawbacks; it does not satisfy
all judgemental equalites of type theory, is cumbersome to work with in practice
(\cite[p. 4]{huber-2016}) and makes equational proofs less reusable since
equational proofs $a \sim_{X} b$ are inherently `local' to the extensional set
$(X , \sim)$.
%
\section{The equality type}
The usual definition of equality in Agda is an inductive data-type with a single
constructor:
%
%% \VerbatimInput{../libs/main.tex}
% \def\verbatim@font{xits}
\begin{verbatim}
data _≡_ {a} {A : Set a} (x : A) : A → Set a where
  instance refl : x ≡ x
\end{verbatim}
%
I shall refer to this as the (usual) inductive equality type.
