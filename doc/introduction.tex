\chapter{Introduction}
This thesis is a case-study in the application of Cubical Agda in the
context of category theory. At the center of this is the notion of
\nomenindex{equality}. In type-theory there are two pervasive notions
of equality: \nomenindex{judgmental equality} and
\nomenindex{propositional equality}. Judgmental equality is a property
of the type system, it is a property that is automatically checked by
a type checker. As such there are some properties judgmental
equalities must crucially have. It must be \nomenindex{decidable},
\nomenindex{sound}, enjoy \nomenindex{canonicity} and be a
\nomen{congruence relation}. Being decidable simply means that that an
algorithm exists to decide whether two terms are equal. For any
practical implementation the decidability must also be effectively
computable. Soundness means that things judged to be equal actually
\emph{are} considered equal. It must be a congruence relation because
otherwise the relation certainly does not adhere to our notion of
equality. One would be able to conclude things like: $x \nequiv y
\rightarrow f\ x \equiv f\ y$. Canonicity will be explained later in
this introduction after we've seen an example of judgmental- and
propositional equality at play for a simple example.\TODO{How to
  motivate canonicity for equality}.

For propositional equality the decidability requirement is relaxed. It
is not in general possible to decide the correctness of logical
propositions (cf. Hilbert's \nomen{entscheidigungsproblem}).
Propositional equality are provided by the developer. When introducing
definitions this report will use the notation $\defeq$. Judgmental
equalities written $=$. For propositional equalities the notation
$\equiv$ is used.

The usual notion of propositional equality in \nomen{Intensional Type
  Theory} (ITT) is quite restrictive. In the next section a few
motivating examples will highlight this. There exist techniques to
circumvent these problems, as we shall see. This thesis will explore
an extension to Agda that redefines the notion of propositional
equality and as such is an alternative to these other techniques.
%
\section{Motivating examples}
%
In the following two sections I present two examples that illustrate
some limitations inherent in ITT and -- by extension -- Agda.
%
\subsection{Functional extensionality}
\label{sec:functional-extensionality}%
Consider the functions:
%
\begin{multicols}{2}
  \noindent%
  \begin{equation*}%
    f \defeq \lambda\ (n \tp \bN) \to (0 + n \tp \bN)
  \end{equation*}%
  \begin{equation*}%
    g \defeq \lambda\ (n \tp \bN) \to (n + 0 \tp \bN)
  \end{equation*}%
\end{multicols}%
%
The term $n + 0$ is
\nomenindex{definitionally} equal to $n$, which we
write as $n + 0 = n$. This is also called
\nomenindex{judgmental equality}.
We call it definitional equality because the \emph{equality} arises
from the \emph{definition} of $+$ which is:
%
\begin{align*}
  +           & \tp \bN \to \bN \to \bN      \\
  n + 0       & \defeq n                   \\
  n + (\suc{m}) & \defeq \suc{(n + m)}
\end{align*}
%
Note that $0 + n$ is \emph{not} definitionally equal to $n$. $0 + n$
is in normal form. I.e.; there is no rule for $+$ whose left hand side
matches this expression. We \emph{do}, however, have that they are
\nomen{propositionally}{propositional equality} equal, which we write
as $n + 0 \equiv n$. Propositional equality means that there is a
proof that exhibits this relation. Since equality is a transitive
relation we have that $n + 0 \equiv 0 + n$.

Unfortunately we don't have $f \equiv g$. There is no way to construct
a proof asserting the obvious equivalence of $f$ and $g$. Actually
showing this is outside the scope of this text. Essentially it would
involve giving a model for our type theory that validates all our
axioms but where $f \equiv g$ is not true. We cannot show that they
are equal, even though we can prove them equal for all points. For
functions this is exactly the notion of equality that we are
interested in: Functions are considered equal when they are equal for
all inputs. This is called \nomenindex{point wise equality}, where the
\emph{points} of a function refer to its arguments.

%% In the context of category theory functional extensionality is e.g.
%% needed to show that representable functors are indeed functors. The
%% representable functor is defined for a fixed category $\bC$ and an
%% object $X \in \bC$. It's map on objects is defined thus:
%% %
%% \begin{align*}
%% \lambda\ A \to \Arrow\ X\ A
%% \end{align*}
%% %
%% That is, it maps objects to arrows. So, it's map on arrows must map an arrow $\Arrow\ A\ B$ to an
%% The map on objects is defined thus:
%% %
%% \begin{align*}
%% \lambda f \to
%% \end{align*}
%% %
%% The proof obligation that this satisfies the identity law of functors
%% ($\fmap\ \idFun \equiv \idFun$) thus becomes:
%% %
%% \begin{align*}
%% \Hom(A, \idFun_{\bX}) = (\lambda\ g \to \idFun \comp g) \equiv \idFun_{\Sets}
%% \end{align*}
%% %
%% One needs functional extensionality to ``go under'' the function arrow and apply
%% the (left) identity law of the underlying category to prove $\idFun \comp g
%% \equiv g$ and thus close the goal.
%
\subsection{Equality of isomorphic types}
%
Let $\top$ denote the unit type -- a type with a single constructor.
In the propositions as types interpretation of type theory $\top$ is
the proposition that is always true. The type $A \x \top$ and $A$ has
an element for each $a \tp A$. So in a sense they have the same shape
(Greek;
\nomenindex{isomorphic}). The second element of the pair does not
add any ``interesting information''. It can be useful to identify such
types. In fact, it is quite commonplace in mathematics. Say we look at
a set $\{x \mid \phi\ x \land \psi\ x\}$ and somehow conclude that
$\psi\ x \equiv \top$ for all $x$. A mathematician would immediately
conclude $\{x \mid \phi\ x \land \psi\ x\} \equiv \{x \mid \phi\ x\}$
without thinking twice. Unfortunately such an identification can not
be performed in ITT.

More specifically what we are interested in is a way of identifying

\nomenindex{equivalent} types. I will return to the definition of
equivalence later in section \S\ref{sec:equiv}, but for now it is
sufficient to think of an equivalence as a one-to-one correspondence.
We write $A \simeq B$ to assert that $A$ and $B$ are equivalent types.
The principle of univalence says that:
%
$$\mathit{univalence} \tp (A \simeq B) \simeq (A \equiv B)$$
%
In particular this allows us to construct an equality from an equivalence
($\mathit{ua} \tp (A \simeq B) \to (A \equiv B)$) and vice versa.

\section{Formalizing Category Theory}
%
The above examples serve to illustrate a limitation of ITT. One case where these
limitations are particularly prohibitive is in the study of Category Theory. At
a glance category theory can be described as ``the mathematical study of
(abstract) algebras of functions'' (\cite{awodey-2006}). By that token
functional extensionality is particularly useful for formulating Category
Theory. In Category theory it is also common to identify isomorphic structures
and univalence gives us a way to make this notion precise. In fact we can
formulate this requirement within our formulation of categories by requiring the
\emph{categories} themselves to be univalent as we shall see.

\section{Context}
\label{sec:context}
%
The idea of formalizing Category Theory in proof assistants is not new. There
are a multitude of these available online. Just as a first reference see this
question on Math Overflow: \cite{mo-formalizations}. Notably these
implementations of category theory in Agda:
%
\begin{itemize}
\item
  A formalization in Agda using the setoid approach:
  \url{https://github.com/copumpkin/categories}
\item
  A formalization in Agda with univalence and functional
  extensionality as postulates:
  \url{https://github.com/pcapriotti/agda-categories}
\item
  A formalization in Coq in the homotopic setting:
  \url{https://github.com/HoTT/HoTT/tree/master/theories/Categories}
\item
  A formalization in CubicalTT - a language designed for cubical type theory.
  Formalizes many different things, but only a few concepts from category
  theory:
  \url{https://github.com/mortberg/cubicaltt}
\end{itemize}
%
The contribution of this thesis is to explore how working in a cubical setting
will make it possible to prove more things and to reuse proofs and to try and
compare some aspects of this formalization with the existing ones.\TODO{How can
  I live up to this?}

There are alternative approaches to working in a cubical setting where
one can still have univalence and functional extensionality. One
option is to postulate these as axioms. This approach, however, has
other shortcomings, e.g. you lose \nomenindex{canonicity}
(\TODO{Pageno!} \cite{huber-2016}). Canonicity means that any well
typed term evaluates to a \emph{canonical} form. For example for a
closed term $e \tp \bN$ it will be the case that $e$ reduces to $n$
applications of $\mathit{suc}$ to $0$ for some $n$; $e =
\mathit{suc}^n\ 0$. Without canonicity terms in the language can get
``stuck'' -- meaning that they do not reduce to a canonical form.

Another approach is to use the \emph{setoid interpretation} of type
theory (\cite{hofmann-1995,huber-2016}). With this approach one works
with
\nomenindex{extensional sets} $(X, \sim)$, that is a type $X \tp \MCU$
and an equivalence relation $\sim\ \tp X \to X \to \MCU$ on that type.
Under the setoid interpretation the equivalence relation serve as a
sort of ``local'' propositional equality. Since the developer gets to
pick this relation it is not guaranteed to be a congruence relation
apriori. So this must be verified manually by the developer.
Furthermore, functions between different setoids must be shown to be
setoid homomorphism, that is; they preserve the relation.

This approach has other drawbacks; it does not satisfy
all propositional equalities of type theory (\TODO{Citation needed}), is
cumbersome to work with in practice (\cite[p. 4]{huber-2016}) and makes
equational proofs less reusable since equational proofs $a \sim_{X} b$ are
inherently `local' to the extensional set $(X , \sim)$.

\section{Conventions}
\TODO{Talk a bit about terminology. Find a good place to stuff this little
  section.}

In the remainder of this paper I will use the term
\nomenindex{Type} to describe --
well, types. Thereby diverging from the notation in Agda where the keyword
\texttt{Set} refers to types.
\nomenindex{Set} on the other hand shall refer to the
homotopical notion of a set. I will also leave all universe levels implicit.

And I use the term
\nomenindex{arrow} to refer to morphisms in a category,
whereas the terms
\nomenindex{morphism},
\nomenindex{map} or
\nomenindex{function}
shall be reserved for talking about type theoretic functions; i.e.
functions in Agda.

$\defeq$ will be used for introducing definitions. $=$ will be used to for
judgmental equality and $\equiv$ will be used for propositional equality.

All this is summarized in the following table:

\begin{center}
\begin{tabular}{ c c c }
Name & Agda & Notation \\
\hline

\varindex{Type}            & \texttt{Set}         & $\Type$            \\

\varindex{Set}             & \texttt{Σ Set IsSet} & $\Set$             \\
Function, morphism, map & \texttt{A → B}       & $A → B$            \\
Dependent- ditto        & \texttt{(a : A) → B} & $∏_{a \tp A} B$  \\

\varindex{Arrow}           & \texttt{Arrow A B}   & $\Arrow\ A\ B$     \\

\varindex{Object}          & \texttt{C.Object}    & $̱ℂ.Object$     \\
Definition              & \texttt{=}           & $̱\defeq$       \\
Judgmental equality     & \null                & $̱=$            \\
Propositional equality  & \null                & $̱\equiv$
\end{tabular}
\end{center}
