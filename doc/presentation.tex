\documentclass[a4paper,handout]{beamer}
\usepackage[utf8]{inputenc}

\usepackage{natbib}
\usepackage[
  hidelinks,
  pdfusetitle,
  pdfsubject={category theory},
  pdfkeywords={type theory, homotopy theory, category theory, agda}]
  {hyperref}

\usepackage{graphicx}

\usepackage{parskip}
\usepackage{multicol}
\usepackage{amssymb,amsmath,amsthm,stmaryrd,mathrsfs,wasysym}
\usepackage[toc,page]{appendix}
\usepackage{xspace}
%% \usepackage{geometry}

% \setlength{\parskip}{10pt}

% \usepackage{tikz}
% \usetikzlibrary{arrows, decorations.markings}

% \usepackage{chngcntr}
% \counterwithout{figure}{section}

\usepackage{listings}
\usepackage{fancyvrb}

\usepackage{mathpazo}
\usepackage[scaled=0.95]{helvet}
\usepackage{courier}
\linespread{1.05} % Palatino looks better with this

\usepackage{lmodern}

\usepackage{fontspec}
\usepackage{sourcecodepro}
%% \setmonofont{Latin Modern Mono}
%% \setmonofont[Mapping=tex-text]{FreeMono.otf}
%% \setmonofont{FreeMono.otf}


\pagestyle{fancyplain}
\setlength{\headheight}{15pt}
\renewcommand{\chaptermark}[1]{\markboth{\textsc{Chapter \thechapter. #1}}{}}
\renewcommand{\sectionmark}[1]{\markright{\textsc{\thesection\ #1}}}

% Allows for the use of unicode-letters:
\usepackage{unicode-math}


%% \RequirePackage{kvoptions}

\newcommand{\subsubsubsection}[1]{\textbf{#1}}
\newcommand{\WIP}{\textbf{WIP}}

\newcommand{\coloneqq}{\mathrel{\vcenter{\baselineskip0.5ex \lineskiplimit0pt
                     \hbox{\scriptsize.}\hbox{\scriptsize.}}}%
  =}

\newcommand{\defeq}{\triangleq}
%% Alternatively:
%% \newcommand{\defeq}{≔}
\newcommand{\bN}{\mathbb{N}}
\newcommand{\bC}{\mathbb{C}}
\newcommand{\bX}{\mathbb{X}}
% \newcommand{\to}{\rightarrow}
\newcommand{\mto}{\mapsto}
\newcommand{\UU}{\ensuremath{\mathcal{U}}\xspace}
\let\type\UU
\newcommand{\MCU}{\UU}
\newcommand{\nomen}[1]{\emph{#1}}
\newcommand{\todo}[1]{\textit{#1}}
\newcommand{\comp}{\circ}
\newcommand{\x}{\times}
\newcommand\inv[1]{#1\raisebox{1.15ex}{$\scriptscriptstyle-\!1$}}
\newcommand{\tp}{\;\mathord{:}\;}
\newcommand{\Type}{\mathcal{U}}

\usepackage{graphicx}
\makeatletter
\newcommand{\shorteq}{%
  \settowidth{\@tempdima}{-}% Width of hyphen
  \resizebox{\@tempdima}{\height}{=}%
}
\makeatother
\newcommand{\var}[1]{\ensuremath{\mathit{#1}}}
\newcommand{\Hom}{\var{Hom}}
\newcommand{\fmap}{\var{fmap}}
\newcommand{\bind}{\var{bind}}
\newcommand{\join}{\var{join}}
\newcommand{\omap}{\var{omap}}
\newcommand{\pure}{\var{pure}}
\newcommand{\idFun}{\var{id}}
\newcommand{\Sets}{\var{Sets}}
\newcommand{\Set}{\var{Set}}
\newcommand{\hSet}{\var{hSet}}
\newcommand{\id}{\var{id}}
\newcommand{\isEquiv}{\var{isEquiv}}
\newcommand{\idToIso}{\var{idToIso}}
\newcommand{\isSet}{\var{isSet}}
\newcommand{\isContr}{\var{isContr}}
\newcommand\Object{\var{Object}}
\newcommand\Functor{\var{Functor}}
\newcommand\isProp{\var{isProp}}
\newcommand\propPi{\var{propPi}}
\newcommand\propSig{\var{propSig}}
\newcommand\PreCategory{\var{PreCategory}}
\newcommand\IsPreCategory{\var{IsPreCategory}}
\newcommand\isIdentity{\var{isIdentity}}
\newcommand\propIsIdentity{\var{propIsIdentity}}
\newcommand\IsCategory{\var{IsCategory}}
\newcommand\Gl{\var{\lambda}}
\newcommand\lemPropF{\var{lemPropF}}
\newcommand\isPreCategory{\var{isPreCategory}}
\newcommand\congruence{\var{cong}}
\newcommand\identity{\var{identity}}
\newcommand\isequiv{\var{isequiv}}
\newcommand\qinv{\var{qinv}}
\newcommand\fiber{\var{fiber}}
\newcommand\shuffle{\var{shuffle}}
\newcommand\Univalent{\var{Univalent}}
\newcommand\refl{\var{refl}}
\newcommand\isoToId{\var{isoToId}}
\newcommand\rrr{\ggg}
\newcommand\fish{\mathrel{\wideoverbar{\rrr}}}
\newcommand\fst{\var{fst}}
\newcommand\snd{\var{snd}}
\newcommand\Path{\var{Path}}
\newcommand\Category{\var{Category}}
\newcommand\TODO[1]{TODO: \emph{#1}}
\newcommand*{\QED}{\hfill\ensuremath{\square}}%
\newcommand\uexists{\exists!}
\newcommand\Arrow{\var{Arrow}}
\newcommand\NTsym{\var{NT}}
\newcommand\NT[2]{\NTsym\ #1\ #2}
\newcommand\Endo[1]{\var{Endo}\ #1}
\newcommand\EndoR{\mathcal{R}}

\title{Univalent Categories}
\author{Frederik Hangh{\o}j Iversen}
\institute{Chalmers University of Technology}
\begin{document}
\frame{\titlepage}

\begin{frame}
  \frametitle{Motivating example}
  \framesubtitle{Functional extensionality}
Consider the functions
\begin{align*}
  \var{zeroLeft} & \defeq (n \tp \bN) \mto (0 + n \tp \bN) \\
  \var{zeroRight} & \defeq (n \tp \bN) \mto (n + 0 \tp \bN)
\end{align*}
\pause
We have
%
$$
\prod_{n \tp \bN} n + 0 \equiv 0 + n
$$
%
\pause
But not
%
$$
\var{zeroLeft} \equiv \var{zeroRight}
$$
%
\pause
We need
%
$$
\funExt \tp \prod_{a \tp A} f\ a \equiv g\ a \to f \equiv g
$$

\end{frame}
\begin{frame}
  \frametitle{Motivating example}
  \framesubtitle{Univalence}
  Consider the set
  $\{x \mid \phi\ x \land \psi\ x\}$
  \pause

  If we show $\forall x . \psi\ x \equiv \top$
  then we want to conclude
  $\{x \mid \phi\ x \land \psi\ x\} \equiv \{x \mid \phi\ x\}$
  \pause

  We need univalence:
  $$(A \simeq B) \simeq (A \equiv B)$$
  \pause
%
  We will return to $\simeq$, but for not, think of it as an
  isomorphism, so it induces maps:
  \begin{align*}
    \var{toPath}  & \tp (A \simeq B) \to (A \equiv B) \\
    \var{toEquiv} & \tp (A \equiv B) \to (A \simeq B)
  \end{align*}
\end{frame}
\begin{frame}
  \frametitle{Paths}
  \framesubtitle{Definition}
Heterogeneous paths
\begin{equation*}
  \Path \tp (P \tp I → \MCU) → P\ 0 → P\ 1 → \MCU
\end{equation*}
\pause
  For $P \tp I \to \MCU$, $A \tp \MCU$ and $a_0, a_1 \tp A$
  inhabitants of $\Path\ P\ a_0\ a_1$ are like functions
%
$$
p \tp \prod_{i \tp I} P\ i
$$
%
Which satisfy $p\ 0 & = a_0$ and $p\ 1 & = a_1$
\pause

Homogenous paths
$$
a_0 \equiv a_1 \defeq \Path\ (\var{const}\ A)\ a_0\ a_1
$$
\end{frame}
\begin{frame}
\frametitle{Paths}
\framesubtitle{Functional extenstionality}
$$
\funExt & \tp \prod_{a \tp A} f\ a \equiv g\ a \to f \equiv g
$$
\pause
$$
\funExt\ p \defeq λ i\ a → p\ a\ i
$$
\pause
$$
\funExt\ (\var{const}\ \refl)
\tp
\var{zeroLeft} \equiv \var{zeroRight}
$$
\end{frame}
\begin{frame}
  \frametitle{Paths}
  \framesubtitle{Homotopy levels}
\begin{align*}
& \isContr    && \tp    \MCU \to \MCU \\
& \isContr\ A && \defeq \sum_{c \tp A} \prod_{a \tp A} a \equiv c
\end{align*}
\pause
\begin{align*}
& \isProp    && \tp \MCU \to \MCU \\
& \isProp\ A && \defeq \prod_{a_0, a_1 \tp A} a_0 \equiv a_1
\end{align*}
\pause
\begin{align*}
& \isSet    && \tp \MCU \to \MCU \\
& \isSet\ A && \defeq \prod_{a_0, a_1 \tp A} \isProp\ (a_0 \equiv a_1)
\end{align*}
\pause
\end{frame}
\begin{frame}
\frametitle{Paths}
\framesubtitle{A few lemmas}
Let $D$ be a type-family:
$$
D \tp \prod_{b \tp A} \prod_{p \tp a ≡ b} \MCU
$$
%
\pause
And $d$ and in inhabitant of $D$ at $\refl$:
%
$$
d \tp D\ a\ \refl
$$
%
\pause
We then have the function:
%
\begin{equation}
\pathJ\ D\ d \tp \prod_{b \tp A} \prod_{p \tp a ≡ b} D\ a\ p
\end{equation}
\end{frame}
\begin{frame}
\frametitle{Paths}
\framesubtitle{A few lemmas}
Given
\begin{align*}
  A           & \tp \MCU \\
  P           & \tp A \to \MCU \\
  \var{propP} & \tp \prod_{x \tp A} \isProp\ (P\ x) \\
  p           & \tp a_0 \equiv a_1 \\
  p_0         & \tp P\ a_0 \\
  p_1         & \tp P\ a_1
\end{align*}
%
We have
$$
\lemPropF\ \var{propP}\ p
\tp
\Path\ (\lambda\; i \mto P\ (p\ i))\ p_0\ p_1
$$
%
\end{frame}
\begin{frame}
\frametitle{Paths}
\framesubtitle{A few lemmas}
$\prod$ preserves $\isProp$:
$$
\mathit{propPi}
\tp
\left(\prod_{a \tp A} \isProp\ (P\ a)\right)
\to \isProp\ \left(\prod_{a \tp A} P\ a\right)
$$
\pause
$\sum$ preserves $\isProp$:
$$
\mathit{propSig} \tp \isProp\ A \to \left(\prod_{a \tp A} \isProp\ (P\ a)\right) \to \isProp\ \left(\sum_{a \tp A} P\ a\right)
$$
\end{frame}
\begin{frame}
\frametitle{Categories}
\framesubtitle{Definition}
Data:
\begin{align*}
  \Object   & \tp \Type \\
  \Arrow    & \tp \Object \to \Object \to \Type \\
  \identity & \tp \Arrow\ A\ A \\
  \lll      & \tp \Arrow\ B\ C \to \Arrow\ A\ B \to \Arrow\ A\ C
\end{align*}
%
Laws:
%
$$
h \lll (g \lll f) ≡ (h \lll g) \lll f
$$
$$
\identity \lll f ≡ f \x
f \lll \identity ≡ f
$$
\pause
1-categories:
$$
\isSet\ (\Arrow\ A\ B)
$$
\pause
Univalent categories:
$$
\isEquiv\ (A \equiv B)\ (A \approxeq B)\ \idToIso
$$
\end{frame}
\begin{frame}
\frametitle{Categories}
\framesubtitle{Univalence}
\begin{align*}
\var{IsIdentity} & \defeq
\prod_{A\ B \tp \Object} \prod_{f \tp \Arrow\ A\ B} \phi\ f
%% \\
%%   & \mathrel{\ } \identity \lll f \equiv f \x f \lll \identity \equiv f
\end{align*}
where
$$
\phi\ f \defeq \identity \lll f \equiv f \x f \lll \identity \equiv f
$$
Let $\approxeq$ denote ismorphism of objects. We can then construct
the identity isomorphism in any category:
$$
\identity , \identity , \var{isIdentity} \tp A \approxeq A
$$
Likewise since paths are substitutive we can promote a path to an isomorphism:
$$
\idToIso \tp A ≡ B → A ≊ B
$$
For a category to be univalent we require this to be an equivalence: 
\end{frame}
\end{document}
