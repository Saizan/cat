\chapter{Category Theory}
\label{ch:implementation}
This implementation formalizes the following concepts:
%
\begin{enumerate}[i.]
\item Categories
\item Functors
\item Products
\item Exponentials
\item Cartesian closed categories
\item Natural transformations
\item Yoneda embedding
\item Monads
\item Categories
  \begin{enumerate}[i.]
  \item Opposite category
  \item Category of sets
  \item ``Pair category''
  \end{enumerate}
\end{enumerate}
%
Furthermore the following items have been partly formalized:
%
\begin{enumerate}[i.]
\item The (higher) category of categories.
\item Category of relations
\item Category of functors and natural transformations -- only as a precategory
\item Free category
\item Monoidal objects
\item Monoidal categories
\end{enumerate}
%
As well as a range of various results about these. E.g. I have shown that the
category of sets has products. In the following I aim to demonstrate some of the
techniques employed in this formalization and in the interest of brevity I will
not detail all the things I have formalized. In stead, I have selected a parts
of this formalization that highlight some interesting proof techniques relevant
to doing proofs in Cubical Agda.

One such technique that is pervasive to this formalization is the idea of
distinguishing types with more or less homotopical structure. To do this I have
followed the following design-principle: I have split concepts up into things
that represent ``data'' and ``laws'' about this data. The idea is that we can
provide a proof that the laws are mere propositions. As an example a category is
defined to have two members: `raw` which is a collection of the data and
`isCategory` which asserts some laws about that data.

This allows me to reason about things in a more ``standard mathematical way'',
where one can reason about two categories by simply focusing on the data. This
is achieved by creating a function embodying the ``equality principle'' for a
given type.

\section{Categories}
The data for a category consist of a type for the sort of objects; a type for
the sort of arrows; an identity arrow and a composition operation for arrows.
Another record encapsulates some laws about this data: associativity of
composition, identity law for the identity morphism. These are standard
constituents of a category and can be found in typical mathematical expositions
on the topic. We, however, impose one further requirement on what it means to be
a category, namely that the type of arrows form a set.

Such categories are called \nomen{1-categories}. It's possible to relax this
requirement. This would lead to the notion of higher categories (\cite[p.
  307]{hott-2013}). For the purpose of this project, however, this report will
restrict itself to 1-categories. Making based on higher categories would be a
very natural possible extension of this work.

Raw categories satisfying all of the above requirements are called a
\nomen{pre}-categories. As a further requirement to be a proper category we
require it to be univalent. Before we can define this, I must introduce two more
definitions: If we let $p$ be a witness to the identity law, which formally is:
%
\begin{equation}
  \label{eq:identity}
  \var{IsIdentity} \defeq
  \prod_{A\ B \tp \Object} \prod_{f \tp A \to B}
    \id \comp f \equiv f \x f \comp \id \equiv f
\end{equation}
%
Then we can construct the identity isomorphism $\var{idIso} \tp \identity,
\identity, p \tp A \approxeq A$ for any object $A$. Here $\approxeq$ denotes
isomorphism on objects (whereas $\cong$ denotes isomorphism of types). This will
be elaborated further on in sections \ref{sec:equiv} and \ref{sec:univalence}.
Moreover, due to substitution for paths we can construct an isomorphism from
\emph{any} path:
%
\begin{equation}
\var{idToIso} : A ≡ B → A ≊ B
\end{equation}
%
The univalence criterion for categories states that this map must be an
equivalence. The requirement is similar to univalence for types, but where
isomorphism on objects play the role of equivalence on types. Formally:
%
\begin{align}
\label{eq:cat-univ}
\isEquiv\ (A \equiv B)\ (A \approxeq B)\ \idToIso
\end{align}
%
Note that \ref{eq:cat-univ} is \emph{not} the same as:
%
\begin{equation}
\label{eq:cat-univalence}
\tag{Univalence, category}
(A \equiv B) \simeq (A \approxeq B)
\end{equation}
%
However the two are logically equivalent: One can construct the latter from the
former simply by ``forgetting'' that $\idToIso$ plays the role of the
equivalence. The other direction is more involved and will be discussed in
section \ref{sec:univalence}.

In summary, the definition of a category is the following collection of data:
%
\begin{align}
  \Object   & \tp \Type \\
  \Arrow    & \tp \Object \to \Object \to \Type \\
  \identity & \tp \Arrow\ A\ A \\
  \lll      & \tp \Arrow\ B\ C \to \Arrow\ A\ B \to \Arrow\ A\ C
\end{align}
%
And laws:
%
\begin{align}
\tag{associativity}
h \lll (g \lll f) ≡ (h \lll g) \lll f \\
\tag{identity}
\identity \lll f ≡ f \x
f \lll \identity ≡ f
\\
\label{eq:arrows-are-sets}
\tag{arrows are sets}
\isSet\ (\Arrow\ A\ B)\\
\tag{\ref{eq:cat-univ}}
\isEquiv\ (A \equiv B)\ (A \approxeq B)\ \idToIso
\end{align}
%
$\lll$ denotes arrow composition (right-to-left), and reverse function
composition (left-to-right, diagrammatic order) is denoted $\rrr$. The objects
($A$, $B$ and $C$) and arrow ($f$, $g$, $h$) are implicitly universally
quantified.

With all this in place it is now possible to prove that all the laws are indeed
mere propositions. Most of the proofs simply use the fact that the type of
arrows are sets. This is because most of the laws are a collection of equations
between arrows in the category. And since such a proof does not have any content
exactly because the type of arrows form a set, two witnesses must be the same.
All the proofs are really quite mechanical. Lets have a look at one of them.
Proving that \ref{eq:identity} is a mere proposition:
%
\begin{equation}
  \isProp\ \var{IsIdentity}
\end{equation}
%
There are multiple ways to prove this. Perhaps one of the more intuitive proofs
is by way of the `combinators' $\propPi$ and $\propSig$ presented in sections
\ref{sec:propPi} and \ref{sec:propSig}:
%
\begin{align*}
\var{propPi} & \tp \left(\prod_{a \tp A} \isProp\ (P\ a)\right) \to \isProp\ \left(\prod_{a \tp A} P\ a\right)
  \\
\var{propSig} & \tp \isProp\ A \to \left(\prod_{a \tp A} \isProp\ (P\ a)\right) \to \isProp\ \left(\sum_{a \tp A} P\ a\right)
\end{align*}
%
So the proof goes like this: We `eliminate' the 3 function abstractions by
applying $\propPi$ three times. So our proof obligation becomes:
%
$$
\isProp \left( \id \comp f \equiv f \x f \comp \id \equiv f \right)
$$
%
Then we eliminate the (non-dependent) sigma-type by applying $\propSig$ giving
us the two obligations: $\isProp\ (\id \comp f \equiv f)$ and $\isProp\ (f \comp
\id \equiv f)$ which follows from the type of arrows being a
set.

This example illustrates nicely how we can use these combinators to reason about
`canonical' types like $\sum$ and $\prod$. Similar combinators can be defined
at the other homotopic levels. These combinators are however not applicable in
situations where we want to reason about other types - e.g. types we've defined
ourselves. For instance, after we've proven that all the projections of
pre-categories are propositions, then we would like to bundle this up to show
that the type of pre-categories is also a proposition. Formally:
%
\begin{equation}
\label{eq:propIsPreCategory}
\isProp\ \IsPreCategory
\end{equation}
%
Where The definition of $\IsPreCategory$ is the triple:
%
\begin{align*}
\var{isAssociative} & \tp \var{IsAssociative}\\
\var{isIdentity}    & \tp \var{IsIdentity}\\
\var{arrowsAreSets} & \tp \var{ArrowsAreSets}
\end{align*}
%
Each corresponding to the first three laws for categories. Note that since
$\IsPreCategory$ is not formulated with a chain of sigma-types we wont have any
combinators available to help us here. In stead we'll have to use the path-type
directly.

\ref{eq:propIsPreCategory} is judgmentally the same as
%
$$
\prod_{a\ b \tp \IsPreCategory} a \equiv b
$$
%
So let $a\ b \tp \IsPreCategory$ be given. To prove the equality $a \equiv b$ is
to give a continuous path from the index-type into the path-space. I.e. a
function $I \to \IsPreCategory$. This path must satisfy being being judgmentally
the same as $a$ at the left endpoint and $b$ at the right endpoint. We know we
can form a continuous path between all projections of $a$ and $b$, this follows
from the type of all the projections being mere propositions. For instance, the
path between $a.\isIdentity$ and $b.\isIdentity$ is simply formed by:
%
$$
\propIsIdentity\ a.\isIdentity\ b.\isIdentity
\tp
a.\isIdentity \equiv b.\isIdentity
$$
%
So to give the continuous function $I \to \IsPreCategory$, which is our goal, we
introduce $i \tp I$ and proceed by constructing an element of $\IsPreCategory$
by using the fact that all the projections are propositions to generate paths
between all projections. Once we have such a path e.g. $p \tp a.\isIdentity
\equiv b.\isIdentity$ we can eliminate it with $i$ and thus obtain $p\ i \tp
(p\ i).\isIdentity$. This element satisfies exactly that it corresponds to the
corresponding projections at either endpoint. Thus the element we construct at
$i$ becomes the triple:
%
\begin{equation}
\label{eq:proof-prop-IsPreCategory}
\begin{aligned}
  & \var{propIsAssociative} && a.\var{isAssociative}\
       && b.\var{isAssociative} && i  \\
  & \var{propIsIdentity}    && a.\var{isIdentity}\
       && b.\var{isIdentity}    && i  \\
  & \var{propArrowsAreSets} && a.\var{arrowsAreSets}\
       && b.\var{arrowsAreSets} && i
\end{aligned}
\end{equation}
%
I've found this to be a general pattern when proving things in homotopy type
theory, namely that you have to wrap and unwrap equalities at different levels.
It is worth noting that proving this theorem with the regular inductive equality
type would already not be possible, since we at least need extensionality (the
projections are all $\prod$-types). Assuming we had functional extensionality
available to us as an axiom, we would use functional extensionality (in
reverse?) to retrieve the equalities in $a$ and $b$, pattern-match on them to
see that they are both $\var{refl}$ and then close the proof with $\var{refl}$.
Of course this theorem is not so interesting in the setting of ITT since we know
a priori that equality proofs are unique.

The situation is a bit more complicated when we have a dependent type. For
instance, when we want to show that $\IsCategory$ is a mere proposition.
$\IsCategory$ is a record with two fields, a witness to being a pre-category and
the univalence condition. Recall that the univalence condition is indexed by the
identity-proof. So to follow the same recipe as above, let $a\ b \tp
\IsCategory$ be given, to show them equal, we now need to give two paths. One homogeneous:
%
$$
p \tp a.\isPreCategory \equiv b.\isPreCategory
$$
%
and one heterogeneous:
%
$$
\Path\ (\lambda\; i \to (p\ i).Univalent)\ a.\isPreCategory\ b.\isPreCategory
$$
%
Which depends on the choice of $p$. The first of these we can provide since, as
we have shown, $\IsPreCategory$ is a proposition. However, even though
$\Univalent$ is also a proposition, we cannot use this directly to show the
latter. This is because $\isProp$ talks about non-dependent paths. So we need to
'promote' the result that univalence is a proposition to a heterogeneous path.
To this end we can use $\lemPropF$, which was introduced in \ref{sec:lemPropF}.

In this case $A = \var{IsIdentity}\ \identity$ and $B = \var{Univalent}$. We've
shown that being a category is a proposition, a result that holds for any choice
of identity proof. Finally we must provide a proof that the identity proofs at
$a$ and $b$ are indeed the same, this we can extract from $p$ by applying
congruence of paths:
%
$$
\congruence\ \var{isIdentity}\ p
$$
%
And this finishes the proof that being-a-category is a mere proposition
(\ref{eq:propIsPreCategory}).

When we have a proper category we can make precise the notion of ``identifying
isomorphic types'' \TODO{cite Awodey here}. That is, we can construct the
function:
%
$$
\isoToId \tp (A \approxeq B) \to (A \equiv B)
$$
%
A perhaps somewhat surprising application of this is that we can show that
terminal objects are propositional:
%
\begin{align}
\label{eq:termProp}
\isProp\ \var{Terminal}
\end{align}
%
It follows from the usual observation that any two terminal objects are
isomorphic - and since categories are univalent, so are they equal. The proof is
omitted here, but the curious reader can check the implementation for the
details. \TODO{The proof is a bit fun, should I include it?}

\section{Equivalences}
\label{sec:equiv}
The usual notion of a function $f \tp A \to B$ having an inverses is:
%
\begin{equation}
\label{eq:isomorphism}
\sum_{g \tp B \to A} f \comp g \equiv \identity_{B} \x g \comp f \equiv \identity_{A}
\end{equation}
%
This is defined in \cite[p. 129]{hott-2013} where it is referred to as the a
``quasi-inverse''. We shall refer to the type \ref{eq:isomorphism} as
$\Isomorphism\ f$. This also gives rise to the following type:
%
\begin{equation}
A \cong B \defeq \sum_{f \tp A \to B} \Isomorphism\ f
\end{equation}
%
At the same place \cite{hott-2013} gives an ``interface'' for what the judgment
$\isEquiv \tp (A \to B) \to \MCU$ must provide:
%
\begin{align}
\var{fromIso}   & \tp \Isomorphism\ f \to \isEquiv\ f \\
\var{toIso}     & \tp \isEquiv\ f \to \Isomorphism\ f \\
\label{eq:propIsEquiv}
                &\mathrel{\ } \isEquiv\ f
\end{align}
%
The maps $\var{fromIso}$ and $\var{toIso}$ naturally extend to these maps:
%
\begin{align}
\var{fromIsomorphism} & \tp A \cong B \to A \simeq B \\
\var{toIsomorphism}   & \tp A \simeq B \to A \cong B
\end{align}
%
Having this interface gives us both: a way to think rather abstractly about how
to work with equivalences and a way to use ad hoc definitions of equivalences.
The specific instantiation of $\isEquiv$ as defined in \cite{cubical-agda} is:
%
$$
isEquiv\ f \defeq \prod_{b \tp B} \isContr\ (\fiber\ f\ b)
$$
where
$$
\fiber\ f\ b \defeq \sum_{a \tp A} \left( b \equiv f\ a \right)
$$
%
I give it's definition here mainly for completeness, because as I stated we can
move away from this specific instantiation and think about it more abstractly
once we have shown that this definition actually works as an equivalence.

$\var{fromIso}$ can be found in \cite{cubical-agda} where it is known as
$\var{gradLemma}$. The implementation of $\var{fromIso}$ as well as the proof
that this equivalence is a proposition (\ref{eq:propIsEquiv}) can be found in my
implementation.

We say that two types $A\;B \tp \Type$ are equivalent exactly if there exists an
equivalence between them:
%
\begin{equation}
\label{eq:equivalence}
A \simeq B \defeq \sum_{f \tp A \to B} \isEquiv\ f
\end{equation}
%
Note that the term equivalence here is overloaded referring both to the map $f
\tp A \to B$ and the type $A \simeq B$. The notion of an isomorphism is
similarly conflated as isomorphism can refer to the type $A \cong B$ as well as
the the map $A \to B$ that witness this. I will use these conflated terms when
it is clear from the context what is being referred to.

Both $\cong$ and $\simeq$ form equivalence relations (no pun intended).

\section{Univalence}
\label{sec:univalence}
As noted in the introduction the univalence for types $A\; B \tp \Type$ states
that:
%
$$
\var{Univalence} \defeq (A \equiv B) \simeq (A \simeq B)
$$
%
As mentioned the univalence criterion for some category $\bC$ says that for all
\emph{objects} $A\;B$ we must have:
$$
\isEquiv\ (A \equiv B)\ (A \approxeq B)\ \idToIso
$$
And I mentioned that this was logically equivalent to
%
$$
(A \equiv B) \simeq (A \approxeq B)
$$
%
Given that we saw in the previous section that we can construct an equivalence
from an isomorphism it suffices to demonstrate:
%
$$
(A \equiv B) \cong (A \approxeq B)
$$
%
That is, we must demonstrate that there is an isomorphism (on types) between
equalities and isomorphisms (on arrows). It's worthwhile to dwell on this for a
few seconds. This type looks very similar to univalence for types and is
therefore perhaps a bit more intuitive to grasp the implications of. Of course
univalence for types (which is a proposition -- i.e. provable) does not imply
univalence of all pre-category since morphisms in a category are not regular
functions -- in stead they can be thought of as a generalization hereof. The univalence criterion therefore is simply a way of restricting arrows
to behave similarly to maps.

I will now mention a few helpful theorems that follow from univalence that will
become useful later.

Obviously univalence gives us an isomorphism between $A \equiv B$ and $A
\approxeq B$. I will name these for convenience:
%
$$
\idToIso \tp A \equiv B \to A \approxeq B
$$
%
$$
\isoToId \tp A \approxeq B \to A \equiv B
$$
%
The next few theorems are variations on theorem 9.1.9 from \cite{hott-2013}. Let
an isomorphism $A \approxeq B$ in some category $\bC$ be given. Name the
isomorphism $\iota \tp A \to B$ and its inverse $\inv{\iota} \tp B \to A$.
Since $\bC$ is a category (and therefore univalent) the isomorphism induces a
path $p \tp A \equiv B$. From this equality we can get two further paths:
$p_{\var{dom}} \tp \var{Arrow}\ A\ X \equiv \var{Arrow}\ B\ X$ and
$p_{\var{cod}} \tp \var{Arrow}\ X\ A \equiv \var{Arrow}\ X\ B$. We
then have the following two theorems:
%
\begin{align}
\label{eq:coeDom}
\var{coeDom} & \tp \prod_{f \tp A \to X}
\var{coe}\ p_{\var{dom}}\ f \equiv f \lll \inv{\iota}
\\
\label{eq:coeCod}
\var{coeCod} & \tp \prod_{f \tp A \to X}
\var{coe}\ p_{\var{cod}}\ f \equiv \iota \lll f
\end{align}
%
I will give the proof of the first theorem here, the second one is analogous.
%
\begin{align*}
\var{coe}\ p_{\var{dom}}\ f
  & \equiv f \lll \inv{(\var{idToIso}\ p)} && \text{lemma} \\
  & \equiv f \lll \inv{\iota}
    && \text{$\var{idToIso}$ and $\var{isoToId}$ are inverses}\\
\end{align*}
%
In the second step we use the fact that $p$ is constructed from the isomorphism
$\iota$ -- $\inv{(\var{idToIso}\ p)}$ denotes the map $B \to A$ induced by the
isomorphism $\var{idToIso}\ p \tp A \cong B$. The helper-lemma is similar to
what we're trying to prove but talks about paths rather than isomorphisms:
%
\begin{equation}
\label{eq:coeDomIso}
\prod_{f \tp \var{Arrow}\ A\ B} \prod_{p \tp A \equiv B}
\var{coe}\ p_{\var{dom}}\ f \equiv f \lll \inv{(\var{idToIso}\ p)}
\end{equation}
%
Again $p_{\var{dom}}$ denotes the path $\var{Arrow}\ A\ X \equiv
\var{Arrow}\ B\ X$ induced by $p$. To prove this statement I let $f$ and $p$
be given and then invoke based-path-induction. The induction will be based at $A
\tp \var{Object}$, so let $\widetilde{B} \tp \Object$ and $\widetilde{p} \tp
A \equiv \widetilde{B}$ be given. The family that we perform induction over will
be:
%
$$
\var{coe}\ {\widetilde{p}}^*\ f
\equiv
f \lll \inv{(\var{idToIso}\ \widetilde{p})}
$$
The base-case therefore becomes:
\begin{align*}
\var{coe}\ {\widetilde{\refl}}^*\ f
& \equiv f \\
& \equiv f \lll \var{identity} \\
& \equiv f \lll \inv{(\var{idToIso}\ \widetilde{\refl})}
\end{align*}
%
The first step follows because reflexivity is a neutral element for coercion.
The second step is the identity law in the category. The last step has to do
with the fact that $\var{idToIso}$ is constructed by substituting according to
the supplied path and since reflexivity is also the neutral element for
substitutions we arrive at the desired expression. To close the
based-path-induction we must supply the value ``at the other''. In this case
this is simply $B \tp \Object$ and $p \tp A \equiv B$ which we have.

And this finishes the proof of \ref{eq:coeDomIso} and thus \ref{eq:coeDom}.
%
\section{Categories}
\subsection{Opposite category}
\label{op-cat}
The first category I'll present is a pure construction on categories. Given some
category we can construct it's dual, called the opposite category. Starting with
a simple example allows us to focus on how we work with equivalences and
univalence in a very simple category where the structure of the category is
rather simple.

Let $\bC$ be some category, we then define the opposite category
$\bC^{\var{Op}}$. It has the same objects, but the type of arrows are flipped,
that is to say an arrow from $A$ to $B$ in the opposite category corresponds to
an arrow from $B$ to $A$ in the underlying category. The identity arrow is the
same as the one in the underlying category (they have the same type). Function
composition will be reverse function composition from the underlying category.

I'll refer to things in terms of the underlying category, unless they have an
over-bar. So e.g. $\idToIso$ is a function in the underlying category and the
corresponding thing is denoted $\wideoverbar{\idToIso}$ in the opposite
category.

Showing that this forms a pre-category is rather straightforward. 
%
$$
h \rrr (g \rrr f) \equiv h \rrr g \rrr f
$$
%
Since $\rrr$ is reverse function composition this is just the symmetric version
of associativity.
%
$$
\var{identity} \rrr f \equiv f \x f \rrr identity \equiv f
$$
%
This is just the swapped version of identity.

Finally, that the arrows form sets just follows by flipping the order of the
arguments. Or in other words; since $\Arrow\ A\ B$ is a set for all $A\;B \tp
\Object$ then so is $Arrow\ B\ A$.

Now, to show that this category is univalent is not as straight-forward. Luckily
section \ref{sec:equiv} gave us some tools to work with equivalences. We saw
that we can prove this category univalent by giving an inverse to
$\wideoverbar{\idToIso} \tp (A \equiv B) \to (A \wideoverbar{\approxeq} B)$.
From the original category we have that $\idToIso \tp (A \equiv B) \to (A \cong
B)$ is an isomorphism. Let us denote it's inverse with $\isoToId \tp (A
\approxeq B) \to (A \equiv B)$. If we squint we can see what we need is a way to
go between $\wideoverbar{\approxeq}$ and $\approxeq$.

An inhabitant of $A \approxeq B$ is simply an arrow $f \tp \var{Arrow}\ A\ B$
and it's inverse $g \tp \var{Arrow}\ B\ A$. In the opposite category $g$ will
play the role of the isomorphism and $f$ will be the inverse. Similarly we can
go in the opposite direction. I name these maps $\var{shuffle} \tp (A \approxeq
B) \to (A \wideoverbar{\approxeq} B)$ and $\var{shuffle}^{-1} \tp (A
\wideoverbar{\approxeq} B) \to (A \approxeq B)$ respectively.

As the inverse of $\wideoverbar{\idToIso}$ I will pick $\wideoverbar{\isoToId}
\defeq \isoToId \comp \var{shuffle}$. The proof that they are inverses go as
follows:
%
\begin{align*}
\wideoverbar{\isoToId} \comp \wideoverbar{\idToIso} & =
\isoToId \comp \var{shuffle} \comp \wideoverbar{\idToIso}
\\
%% ≡⟨ cong (λ φ → φ x) (cong (λ φ → η ⊙ shuffle ⊙ φ) (funExt lem)) ⟩ \\
%
& \equiv
\isoToId \comp \var{shuffle} \comp \inv{\var{shuffle}} \comp \idToIso
&& \text{lemma} \\
%%   ≡⟨⟩ \\
& \equiv
\isoToId \comp \idToIso
&& \text{$\var{shuffle}$ is an isomorphism} \\
& \equiv
\identity
&& \text{$\isoToId$ is an isomorphism}
\end{align*}
%
The other direction is analogous.

The lemma used in step 2 of this proof states that $\wideoverbar{idToIso} \equiv
\inv{\var{shuffle}} \comp \idToIso$. This is a rather straight-forward proof
since being-an-inverse-of is a proposition, so it suffices to show that their
first components are equal, but this holds judgmentally.

This finished the proof that the opposite category is in fact a category. Now,
to prove that that opposite-of is an involution we must show:
%
$$
\prod_{\bC \tp \var{Category}} \left(\bC^{\var{Op}}\right)^{\var{Op}} \equiv \bC
$$
%
As we've seen the laws in $\left(\bC^{\var{Op}}\right)^{\var{Op}}$ get quite
involved.\footnote{We haven't even seen the full story because we've used this
  `interface' for equivalences.} Luckily since being-a-category is a mere
proposition, we need not concern ourselves with this bit when proving the above.
We can use the equality principle for categories that let us prove an equality
just by giving an equality on the data-part. So, given a category $\bC$ all we
must provide is the following proof:
%
$$
\var{raw}\ \left(\bC^{\var{Op}}\right)^{\var{Op}} \equiv \var{raw}\ \bC
$$
%
And these are judgmentally the same. I remind the reader that the left-hand side
is constructed by flipping the arrows, which judgmentally is an involution.

\subsection{Category of sets}
The category of sets has as objects, not types, but only those types that are
homotopic sets. This is encapsulated in Agda with the following type:
%
$$\Set \defeq \sum_{A \tp \MCU} \isSet\ A$$
%
The more straight-forward notion of a category where the objects are types is
not a valid \mbox{(1-)category}. This stems from the fact that types in cubical
Agda types can have higher homotopic structure.

Univalence does not follow immediately from univalence for types:
%
$$(A \equiv B) \simeq (A \simeq B)$$
%
Because here $A\ B \tp \Type$ whereas the objects in this category have the type
$\Set$ so we cannot form the type $\var{hA} \simeq \var{hB}$ for objects
$\var{hA}\;\var{hB} \tp \Set$. In stead I show that this category
satisfies:
%
$$
(\var{hA} \equiv \var{hB}) \simeq (\var{hA} \approxeq \var{hB})
$$
%
Which, as we saw in section \ref{sec:univalence}, is sufficient to show that the
category is univalent. The way that I have shown this is with a three-step
process. For objects $(A, s_A)\; (B, s_B) \tp \Set$ I show the following chain
of equivalences:
%
\begin{align*}
((A, s_A) \equiv (B, s_B))
 & \simeq (A \equiv B) && \ref{eq:equivPropSig} \\
 & \simeq (A \simeq B) && \text{Univalence} \\
 & \simeq ((A, s_A) \approxeq (B, s_B)) && \text{\ref{eq:equivSig} and \ref{eq:equivIso}}
\end{align*}

And since $\simeq$ is an equivalence relation we can chain these equivalences
together. Step one will be proven with the following lemma:
%
\begin{align}
  \label{eq:equivPropSig}
\left(\prod_{a \tp A} \isProp (P\ a)\right) \to \prod_{x\;y \tp \sum_{a \tp A} P\ a} (x \equiv y) \simeq (\fst\ x \equiv \fst\ y)
\end{align}
%
The lemma states that for pairs whose second component are mere propositions
equality is equivalent to equality of the first components. In this case the
type-family $P$ is $\isSet$ which itself is a proposition for any type $A \tp
\Type$. Step two is univalence. Step three will be proven with the following
lemma:
%
\begin{align}
  \label{eq:equivSig}
\prod_{a \tp A} \left( P\ a \simeq Q\ a \right) \to \sum_{a \tp A} P\ a \simeq \sum_{a \tp A} Q\ a
\end{align}
%
Which says that if two type-families are equivalent at all points, then pairs
with identical first components and these families as second components will
also be equivalent. For our purposes $P \defeq \isEquiv\ A\ B$ and $Q \defeq
\var{Isomorphism}$. So we must finally prove:
%
\begin{align}
  \label{eq:equivIso}
\prod_{f \tp A \to B} \left( \isEquiv\ A\ B\ f \simeq \var{Isomorphism}\ f \right)
\end{align}

First, lets prove \ref{eq:equivPropSig}: Let $propP \tp \prod_{a \tp A} \isProp (P\ a)$ and $x\;y \tp \sum_{a \tp A} P\ a$ be given. Because
of $\var{fromIsomorphism}$ it suffices to give an isomorphism between
$x \equiv y$ and $\fst\ x \equiv \fst\ y$:
%
%% FIXME: Too much alignement?
\begin{equation*}
\begin{aligned}
  f & \defeq \congruence\ \fst
    && \tp x       \equiv y       && \to \fst\ x \equiv \fst\ y \\
  g & \defeq \var{lemSig}\ \var{propP}\ x\ y
    && \tp \fst\ x \equiv \fst\ y && \to x       \equiv y
\end{aligned}
\end{equation*}
%
\TODO{Is it confusing that I use point-free style here?} Here $\var{lemSig}$ is
a lemma that says that if the second component of a pair is a proposition, it
suffices to give a path between its first components to construct an equality of
the two pairs:
%
\begin{align*}
\var{lemSig} \tp \left( \prod_{x \tp A} \isProp\ (B\ x) \right) \to
\prod_{u\; v \tp \sum_{a \tp A} B\ a}
  \left( \fst\ u \equiv \fst\ v \right) \to u \equiv v
\end{align*}
%
The proof that these are indeed inverses has been omitted. \TODO{Do I really
  want to omit it?}\QED

Now to prove \ref{eq:equivSig}: Let $e \tp \prod_{a \tp A} \left( P\ a \simeq
Q\ a \right)$ be given. To prove the equivalence, it suffices to give an
isomorphism between $\sum_{a \tp A} P\ a$ and $\sum_{a \tp A} Q\ a$, but since
they have identical first components it suffices to give an isomorphism between
$P\ a$ and $Q\ a$ for all $a \tp A$. This is exactly what we can get from
the equivalence $e$.\QED

Lastly we prove \ref{eq:equivIso}. Let $f \tp A \to B$ be given. For the maps we
choose:
%
\begin{align*}
\var{toIso}
  & \tp \isEquiv\ f             \to \var{Isomorphism}\ f \\
\var{fromIso}
  & \tp \var{Isomorphism}\ f \to \isEquiv\ f
\end{align*}
%
As mentioned in section \ref{sec:equiv}. These maps are not in general inverses
of each other. In stead, we will use the fact that $A$ and $B$ are sets. The first thing we must prove is:
%
\begin{align*}
  \var{fromIso} \comp \var{toIso} \equiv \identity_{\isEquiv\ f}
\end{align*}
%
For this we can use the fact that being-an-equivalence is a mere proposition.
For the other direction:
%
\begin{align*}
  \var{toIso} \comp \var{fromIso} \equiv \identity_{\var{Isomorphism}\ f}
\end{align*}
%
We will show that $\var{Isomorphism}\ f$ is also a mere proposition. To this
end, let $X\;Y \tp \var{Isomorphism}\ f$ be given. Name the maps $x\;y \tp B
\to A$ respectively. Now, the proof that $X$ and $Y$ are the same is a pair of
paths: $p \tp x \equiv y$ and $\Path\ (\lambda\; i \to
\var{AreInverses}\ f\ (p\ i))\ \mathcal{X}\ \mathcal{Y}$ where $\mathcal{X}$
and $\mathcal{Y}$ denotes the witnesses that $x$ (respectively $y$) is an
inverse to $f$. $p$ is inhabited by:
%
\begin{align*}
  x
  & \equiv x \comp \identity \\
  & \equiv x \comp (f \comp y)
  && \text{$y$ is an inverse to $f$} \\
  & \equiv (x \comp f) \comp y \\
  & \equiv \identity \comp y
  && \text{$x$ is an inverse to $f$} \\
  & \equiv y
\end{align*}
%
For the other (dependent) path we can prove that being-an-inverse-of is a
proposition and then use $\lemPropF$. So we prove the generalization:
%
\begin{align}
\label{eq:propAreInversesGen}
\prod_{g \tp B \to A} \isProp\ (\var{AreInverses}\ f\ g)
\end{align}
%
But $\var{AreInverses}\ f\ g$ is a pair of equations on arrows, so we use
$\propSig$ and the fact that both $A$ and $B$ are sets to close this proof.

\subsection{Category of categories}
Note that this category does in fact not exist. In stead I provide the
definition of the ``raw'' category as well as some of the laws.

Furthermore I provide some helpful lemmas about this raw category. For instance
I have shown what would be the exponential object in such a category.

These lemmas can be used to provide the actual exponential object in a context
where we have a witness to this being a category. This is useful if this library
is later extended to talk about higher categories.

\section{Products}
In the following I'll demonstrate a technique for using categories to prove
properties. The goal in this section is to show that products are propositions:
%
$$
\prod_{\bC \tp \Category} \prod_{A\;B \tp \Object} \isProp\ (\var{Product}\ \bC\ A\ B)
$$
%
Where $\var{Product}\ \bC\ A\ B$ denotes the type of products of objects $A$
and $B$ in the category $\bC$. I do this by constructing a category whose
terminal objects are equivalent to products in $\bC$, and since terminal objects
are propositional in a proper category and equivalences preserve homotopy level,
then we know that products also are propositions. But before we get to that,
let's recall the definition of products.

\subsection{Definition of products}
Given a category $\bC$ and two objects $A$ and $B$ in $\bC$ we define the
product (object) of $A$ and $B$ to be an object $A \x B$ in $\bC$ and two arrows
$\pi_1 \tp A \x B \to A$ and $\pi_2 \tp A \x B \to B$ called the projections of
the product. The projections must satisfy the following property:

For all $X \tp Object$, $f \tp \Arrow\ X\ A$ and $g \tp \Arrow\ X\ B$ we have
that there exists a unique arrow $\pi \tp \Arrow\ X\ (A \x B)$ satisfying
%
\begin{align}
\label{eq:umpProduct}
%% \prod_{X \tp Object} \prod_{f \tp \Arrow\ X\ A} \prod_{g \tp \Arrow\ X\ B}\\
%% \uexists_{f \x g \tp \Arrow\ X\ (A \x B)}
\pi_1 \lll \pi \equiv f \x \pi_2 \lll \pi \equiv g
\end{align}
%
$\pi$ is called the product (arrow) of $f$ and $g$.

\subsection{Pair category}

\newcommand\pairA{\mathcal{A}}
\newcommand\pairB{\mathcal{B}}
Given a base category $\bC$ and two objects in this category $\pairA$ and
$\pairB$ we can construct the ``pair category'': \TODO{This is a working title,
  it's nice to have a name for this thing to refer back to}

The type of objects in this category will be an object in the underlying
category, $X$, and two arrows (also from the underlying category)
$\Arrow\ X\ \pairA$ and $\Arrow\ X\ \pairB$.

\newcommand\pairf{\ensuremath{f}}
\newcommand\pairFst{\mathcal{\pi_1}}
\newcommand\pairSnd{\mathcal{\pi_2}}

An arrow between objects $A ,\ a_0 ,\ a_1$ and $B ,\ b_0 ,\ b_1$ in this
category will consist of an arrow from the underlying category $\pairf \tp
\Arrow\ A\ B$ satisfying:
%
\begin{align}
\label{eq:pairArrowLaw}
b_0 \lll f \equiv a_0 \x
b_1 \lll f \equiv a_1
\end{align}

The identity morphism is the identity morphism from the underlying category.
This choice satisfies \ref{eq:pairArrowLaw} because of the right-identity law
from the underlying category.

For composition of arrows $f \tp \Arrow\ A\ B$ and $g \tp \Arrow\ B\ C$ we
choose $g \lll f$ and we must now verify that it satisfies
\ref{eq:pairArrowLaw}:
%
\begin{align*}
  c_0 \lll (f \lll g)
  & \equiv
  (c_0 \lll f) \lll g
  && \text{Associativity} \\
  & \equiv
  b_0 \lll g
  && \text{$f$ satisfies \ref{eq:pairArrowLaw}} \\
  & \equiv
  a_0
  && \text{$g$ satisfies \ref{eq:pairArrowLaw}} \\
\end{align*}
%
Now we must verify the category-laws. For all the laws we will follow the
pattern of using the law from the underlying category, and that the type of
arrows form a set. For instance, to prove associativity we must prove that
%
\begin{align}
\label{eq:productAssoc}
\overline{h} \lll (\overline{g} \lll \overline{f})
\equiv
(\overline{h} \lll \overline{g}) \lll \overline{f}
\end{align}
%
Here $\lll$ refers to the `embellished' composition and $\overline{f}$,
$\overline{g}$ and $\overline{h}$ are triples consisting of arrows from the
underlying category ($f$, $g$ and $h$) and a pair of witnesses to
\ref{eq:pairArrowLaw}.
%% Luckily those winesses are paths in the hom-set of the
%% underlying category which is a set, so these are mere propositions.
The proof obligations is consists of two things. The first one is:
%
\begin{align}
\label{eq:productAssocUnderlying}
h \lll (g \lll f)
\equiv
(h \lll g) \lll f
\end{align}
%
And the other proof obligation is that the witness to \ref{eq:pairArrowLaw} for
the left-hand-side and the right-hand-side are the same.

The proof of the first goal comes directly from the underlying category. The
type of the second goal is very complicated. I will not write it out in full
here, but it suffices to show the type of the path-space. Note that the arrows
in \ref{eq:productAssoc} are arrows from $\mathcal{A} = (A , a_{\pairA} ,
a_{\pairB})$ to $\mathcal{D} = (D , d_{\pairA} , d_{\pairB})$ where
$a_{\pairA}$, $a_{\pairB}$, $d_{\pairA}$ and $d_{\pairB}$ are arrows in the
underlying category. Given that $p$ is the chosen proof of
\ref{eq:productAssocUnderlying} we then have that the witness to
\ref{eq:pairArrowLaw} vary over the type:
%
\begin{align}
\label{eq:productPath}
λ\ i → d_{\pairA} \lll p\ i ≡ 2 a_{\pairA} × d_{\pairB} \lll p\ i ≡ a_{\pairB}
\end{align}
%
And these paths are in the type of the hom-set of the underlying category, so
they are mere propositions. We cannot apply the fact that arrows in $\bC$ are
sets directly, however, since $\isSet$ only talks about non-dependent paths, in
stead we generalize \ref{eq:productPath} to:
%
\begin{align}
\label{eq:productEqPrinc}
\prod_{f \tp \Arrow\ X\ Y} \isProp\ \left( y_{\pairA} \lll f ≡ x_{\pairA} × y_{\pairB} \lll f ≡ x_{\pairB} \right)
\end{align}
%
For all objects $X , x_{\pairA} , x_{\pairB}$ and $Y , y_{\pairA} , y_{\pairB}$,
but this follows from pairs preserving homotopical structure and arrows in the
underlying category being sets. This gives us an equality principle for arrows
in this category that says that to prove two arrows $f, f_0, f_1$ and $g, g_0,
g_1$ equal it suffices to give a proof that $f$ and $g$ are equal.
%% %
%% $$
%% \prod_{(f, f_0, f_1)\; (g,g_0,g_1) \tp \Arrow\ X\ Y} f \equiv g \to (f, f_0, f_1) \equiv (g,g_0,g_1)
%% $$
%% %
And thus we have proven \ref{eq:productAssoc} simply with
\ref{eq:productAssocUnderlying}.

Now we must prove that arrows form a set:
%
$$
\isSet\ (\Arrow\ \mathcal{X}\ \mathcal{Y})
$$
%
Since pairs preserve homotopical structure this reduces to:
%
$$
\isSet\ (\Arrow_{\bC}\ X\ Y)
$$
%
Which holds. And
%
$$
\prod_{f \tp \Arrow\ X\ Y}
\isSet\ \left( y_{\pairA} \lll f ≡ x_{\pairA}
             × y_{\pairB} \lll f ≡ x_{\pairB}
        \right)
$$
%
This we get from \ref{eq:productEqPrinc} and the fact that homotopical structure
is cumulative.

This finishes the proof that this is a valid pre-category.

\subsubsection{Univalence}
To prove that this is a proper category it must be shown that it is univalent.
That is, for any two objects $\mathcal{X} = (X, x_{\mathcal{A}} , x_{\mathcal{B}})$
and $\mathcal{Y} = Y, y_{\mathcal{A}}, y_{\mathcal{B}}$ I will show:
%
\begin{align}
(\mathcal{X} \equiv \mathcal{Y}) \cong (\mathcal{X} \approxeq \mathcal{Y})
\end{align}

I do this by showing that the following sequence of types are isomorphic.

The first type is:
%
\begin{align}
\label{eq:univ-0}
(X , x_{\mathcal{A}} , x_{\mathcal{B}}) ≡ (Y , y_{\mathcal{A}} , y_{\mathcal{B}})
\end{align}
%
The next types will be the triple:
%
\begin{align}
\label{eq:univ-1}
\begin{split}
p \tp & X \equiv Y \\
& \Path\ (λ i → \Arrow\ (p\ i)\ \mathcal{A})\ x_{\mathcal{A}}\ y_{\mathcal{A}} \\
& \Path\ (λ i → \Arrow\ (p\ i)\ \mathcal{B})\ x_{\mathcal{B}}\ y_{\mathcal{B}}
\end{split}
%% \end{split}
\end{align}

The next type is very similar, but in stead of a path we will have an
isomorphism, and create a path from this:
%
\begin{align}
\label{eq:univ-2}
\begin{split}
\var{iso} \tp & X \cong Y \\
& \Path\ (λ i → \Arrow\ (\widetilde{p}\ i)\ \mathcal{A})\ x_{\mathcal{A}}\ y_{\mathcal{A}} \\
& \Path\ (λ i → \Arrow\ (\widetilde{p}\ i)\ \mathcal{B})\ x_{\mathcal{B}}\ y_{\mathcal{B}}
\end{split}
\end{align}
%
Where $\widetilde{p} \defeq \var{isoToId}\ \var{iso} \tp X \equiv Y$.

Finally we have the type:
%
\begin{align}
\label{eq:univ-3}
(X , x_{\mathcal{A}} , x_{\mathcal{B}}) ≊ (Y , y_{\mathcal{A}} , y_{\mathcal{B}})
\end{align}

\emph{Proposition} \ref{eq:univ-0} is isomorphic to \ref{eq:univ-1}: This is
just an application of the fact that a path between two pairs $a_0, a_1$ and
$b_0, b_1$ corresponds to a pair of paths between $a_0,b_0$ and $a_1,b_1$ (check
the implementation for the details).

\emph{Proposition} \ref{eq:univ-1} is isomorphic to \ref{eq:univ-2}:
\TODO{Super complicated}

\emph{Proposition} \ref{eq:univ-2} is isomorphic to \ref{eq:univ-3}: For this I
will show two corollaries of \ref{eq:coeCod}: For an isomorphism $(\iota,
\inv{\iota}, \var{inv}) \tp A \cong B$, arrows $f \tp \Arrow\ A\ X$, $g \tp
\Arrow\ B\ X$ and a heterogeneous path between them, $q \tp \Path\ (\lambda i
\to p_{\var{dom}}\ i)\ f\ g$, where $p_{\var{dom}} \tp \Arrow\ A\ X \equiv
\Arrow\ B\ X$ is a path induced by $\var{iso}$, we have the following two
results
%
\begin{align}
\label{eq:domain-twist-0}
f & \equiv g \lll \iota \\
\label{eq:domain-twist-1}
g & \equiv f \lll \inv{\iota}
\end{align}
%
Proof: \TODO{\ldots}

Now we can prove the equivalence in the following way: Given $(f, \inv{f},
\var{inv}_f) \tp X \cong Y$ and two heterogeneous paths
%
\begin{align*}
p_{\mathcal{A}} & \tp \Path\ (\lambda i \to p_{\var{dom}}\ i)\ x_{\mathcal{A}}\ y_{\mathcal{A}}\\
%
q_{\mathcal{B}} & \tp \Path\ (\lambda i \to p_{\var{dom}}\ i)\ x_{\mathcal{B}}\ y_{\mathcal{B}}
\end{align*}
%
all as in \ref{eq:univ-2}. I use $p_{\var{dom}}$ here again to mean the path
induced by the isomorphism $f, \inv{f}$. I must now construct an isomorphism
$(X, x_{\mathcal{A}}, x_{\mathcal{B}}) \approxeq (Y, y_{\mathcal{A}}, y_{\mathcal{B}})$
as in \ref{eq:univ-3}. That is, an isomorphism in the present category. I remind
the reader that such a gadget is a triple. The first component shall be:
%
\begin{align}
f \tp \Arrow\ X\ Y
\end{align}
%
To show that this choice fits the bill I must now verify that it satisfies
\ref{eq:pairArrowLaw}, which in this case becomes:
%
\begin{align}
y_{\mathcal{A}} \lll f ≡ x_{\mathcal{A}} × y_{\mathcal{B}} \lll f ≡ x_{\mathcal{B}}
\end{align}
%
Which, since $f$ is an isomorphism and $p_{\mathcal{A}}$ (resp. $p_{\mathcal{B}}$)
is a path varying according to a path constructed from this isomorphism, this is
exactly what \ref{eq:domain-twist-0} gives us.
%
The other direction is quite analogous. We choose $\inv{f}$ as the morphism and
prove that it satisfies \ref{eq:pairArrowLaw} with \ref{eq:domain-twist-1}.

We must now show that this choice of arrows indeed form an isomorphism. Our
equality principle for arrows in this category (\ref{eq:productEqPrinc}) gives
us that it suffices to show that $f$ and $\inv{f}$, this is exactly
$\var{inv}_f$.

This concludes the first direction of the isomorphism that we're constructing.
For the other direction we're given just given the isomorphism
%
$$
(f, \inv{f}, \var{inv}_f)
\tp
(X, x_{\mathcal{A}}, x_{\mathcal{B}}) \approxeq (Y, y_{\mathcal{A}}, y_{\mathcal{B}})
$$
%
Projecting out the first component gives us the isomorphism
%
$$
(\fst\ f, \fst\ \inv{f}, \congruence\ \fst\ \var{inv}_f, \congruence\ \fst\ \var{inv}_{\inv{f}})
\tp X \approxeq Y
$$
%
This gives rise to the following paths:
%
\begin{align}
\begin{split}
\widetilde{p} & \tp X \equiv Y \\
\widetilde{p}_{\mathcal{A}} & \tp \Arrow\ X\ \mathcal{A} \equiv \Arrow\ Y\ \mathcal{A} \\
\widetilde{p}_{\mathcal{B}} & \tp \Arrow\ X\ \mathcal{B} \equiv \Arrow\ Y\ \mathcal{B}
\end{split}
\end{align}
%
It then remains to construct the two paths:
%
\begin{align}
\begin{split}
\label{eq:product-paths}
& \Path\ (λ i → \widetilde{p}_{\mathcal{A}}\ i)\ x_{\mathcal{A}}\ y_{\mathcal{A}}\\
& \Path\ (λ i → \widetilde{p}_{\mathcal{B}}\ i)\ x_{\mathcal{B}}\ y_{\mathcal{B}}
\end{split}
\end{align}
%
This is achieved with the following lemma:
%
\begin{align}
\prod_{a \tp A} \prod_{b \tp B} \prod_{q \tp A \equiv B} \var{coe}\ q\ a ≡ b →
\Path\ (λ i → q\ i)\ a\ b
\end{align}
%
Which is used without proof. See the implementation for the details.

\ref{eq:product-paths} is the proven with the propositions:
%
\begin{align}
\begin{split}
\label{eq:product-paths}
\var{coe}\ \widetilde{p}_{\mathcal{A}}\ x_{\mathcal{A}} ≡ y_{\mathcal{A}}\\
\var{coe}\ \widetilde{p}_{\mathcal{B}}\ x_{\mathcal{B}} ≡ y_{\mathcal{B}}
\end{split}
\end{align}
%
The proof of the first one is:
%
\begin{align*}
  \var{coe}\ \widetilde{p}_{\mathcal{A}}\ x_{\mathcal{A}}
  & ≡ x_{\mathcal{A}} \lll \fst\ \inv{f} && \text{$\var{coeDom}$ and the isomorphism $f, \inv{f}$} \\
  & ≡ y_{\mathcal{A}} && \text{\ref{eq:pairArrowLaw} for $\inv{f}$}
\end{align*}
%
We have now constructed the maps between \ref{eq:univ-0} and \ref{eq:univ-1}. It
remains to show that they are inverses of each other. To cut a long story short,
the proof uses the fact that isomorphism-of is propositional and that arrows (in
both categories) are sets. The reader is referred to the implementation for the
gory details.
%
\subsection{Propositionality of products}
%
Now that we've constructed the ``pair category'' I'll demonstrate how to use
this to prove that products are propositional. I will do this by showing that
terminal objects in this category are equivalent to products:
%
\begin{align}
\var{Terminal} ≃ \var{Product}\ ℂ\ \mathcal{A}\ \mathcal{B}
\end{align}
%
And as always we do this by constructing an isomorphism:
%
In the direction $\var{Terminal} → \var{Product}\ ℂ\ \mathcal{A}\ \mathcal{B}$
we're given a terminal object $X, x_𝒜, x_ℬ$. $X$ Will be the product-object and
$x_𝒜, x_ℬ$ will be the product arrows, so it just remains to verify that this is
indeed a product. That is, for an object $Y$ and two arrows $y_𝒜 \tp
\Arrow\ Y\ 𝒜$, $y_ℬ\ \Arrow\ Y\ ℬ$ we must find a unique arrow $f \tp
\Arrow\ Y\ X$ satisfying:
%
\begin{align}
\label{eq:pairCondRev}
\begin{split}
  x_𝒜 \lll f & ≡ y_𝒜 \\
  x_ℬ \lll f & ≡ y_ℬ
\end{split}
\end{align}
%
Since $X, x_𝒜, x_ℬ$ is a terminal object there is a \emph{unique} arrow from
this object to any other object, so also $Y, y_𝒜, y_ℬ$ in particular (which is
also an object in the pair category). The arrow we will play the role of $f$ and
it immediately satisfies \ref{eq:pairCondRev}. Any other arrow satisfying these
conditions will be equal since $f$ is unique.

For the other direction we are now given a product $X, x_𝒜, x_ℬ$. Again this
will be the terminal object. So now it remains that for any other object there
is a unique arrow from that object into $X, x_𝒜, x_ℬ$. Let $Y, y_𝒜, y_ℬ$ be
another object. As the arrow $\Arrow\ Y\ X$ we choose the product-arrow $y_𝒜 \x
y_ℬ$. Since this is a product-arrow it satisfies \ref{eq:pairCondRev}. Let us
name the witness to this $\phi_{y_𝒜 \x y_ℬ}$. So we have picked as our center of
contraction $y_𝒜 \x y_ℬ , \phi_{y_𝒜 \x y_ℬ}$ we must now show that it is
contractible. So let $f \tp \Arrow\ X\ Y$ and $\phi_f$ be given (here $\phi_f$
is the proof that $f$ satisfies \ref{eq:pairCondRev}). The proof will be a pair
of proofs:
%
\begin{alignat}{3}
  p \tp & \Path\ (\lambda i \to \Arrow\ X\ Y)\quad
    && f\quad          && y_𝒜 \x y_ℬ \\
  & \Path\ (\lambda i \to \Phi\ (p\ i))\quad
    && \phi_f\quad     && \phi_{y_𝒜 \x y_ℬ}
\end{alignat}
%
Here $\Phi$ is given as:
$$
\prod_{f \tp \Arrow\ Y\ X}
  x_𝒜 \lll f ≡ y_𝒜
× x_ℬ \lll f ≡ y_ℬ
$$
%
$p$ follows from the universal property of $y_𝒜 \x y_ℬ$. For the latter we will
again use the same trick we did in \ref{eq:propAreInversesGen} and prove this
more general result:
%
$$
\prod_{f \tp \Arrow\ Y\ X} \isProp\ (
  x_𝒜 \lll f ≡ y_𝒜
× x_ℬ \lll f ≡ y_ℬ
)
$$
%
Which follows from arrows being sets and pairs preserving such. Thus we can
close the final proof with an application of $\lemPropF$.

This concludes the proof $\var{Terminal} ≃
\var{Product}\ ℂ\ \mathcal{A}\ \mathcal{B}$ and since we have that equivalences
preserve homotopic levels along with \ref{eq:termProp} we get our final result.
That in any category:
%
\begin{align}
\prod_{A\ B \tp \Object} \isProp\ (\var{Product}\ \bC\ A\ B)
\end{align}
%
\section{Monads}
In this section I present two formulations of monads. The two representations
are referred to as the monoidal- and Kleisli- representation respectively or
simply monoidal monads and Kleisli monads for short. We then show that the two
formulations are equivalent, which due to univalence gives us a path between the
two types.

Let a category $\bC$ be given. In the remainder of this sections all objects and
arrows will implicitly refer to objects and arrows in this category.
%
\subsection{Monoidal formulation}
The monoidal formulation of monads consists of the following data:
%
\begin{align}
\label{eq:monad-monoidal-data}
\begin{split}
    \EndoR      & \tp \Endo ℂ \\
    \var{pure}  & \tp \NT{\EndoR^0}{\EndoR} \\
    \var{join}  & \tp \NT{\EndoR^2}{\EndoR}
\end{split}
\end{align}
%
Here $\NTsym$ denotes natural transformations, the super-script in $\EndoR^2$
Denotes the composition of $\EndoR$ with itself. By the same token $\EndoR^0$ is
a curious way of denoting the identity functor. This notation has been chosen
for didactic purposes.

Denote the arrow-map of $\EndoR$ as $\fmap$, then this data must satisfy the
following laws:
%
\begin{align}
\label{eq:monad-monoidal-laws}
\begin{split}
  \var{join} \lll \fmap\ \var{join}
    & ≡ \var{join} \lll \var{join}\ \fmap \\
  \var{join} \lll \var{pure}\ \fmap     & ≡ \identity \\
  \var{join} \lll \fmap\     \var{pure} & ≡ \identity
\end{split}
\end{align}
%
The implicit arguments to the arrows above have been left out and the objects
they range over are universally quantified.

\subsection{Kleisli formulation}
%
The Kleisli-formulation consists of the following data:
%
\begin{align}
\label{eq:monad-kleisli-data}
\begin{split}
    \EndoR & \tp \Object → \Object \\
    \pure  & \tp % \prod_{X \tp Object}
      \Arrow\ X\ (\EndoR\ X) \\
    \bind  & \tp % \prod_{X\;Y \tp Object} → \Arrow\ X\ (\EndoR\ Y)
      \Arrow\ (\EndoR\ X)\ (\EndoR\ Y)
\end{split}
\end{align}
%
The objects $X$ and $Y$ are implicitly universally quantified.

It's interesting to note here that this formulation does not talk about natural
transformations or other such constructs from category theory. All we have here
is a regular maps on objects and a pair of arrows.
%
This data must satisfy:
%
\begin{align}
\label{eq:monad-monoidal-laws}
\begin{split}
  \bind\ \pure & ≡ \identity_{\EndoR\ X}
  \\
  % \prod_{f \tp \Arrow\ X\ (\EndoR\ Y)}
    \pure \fish f & ≡ f
    \\
  % \prod_{\substack{g \tp \Arrow\ Y\ (\EndoR\ Z)\\f \tp \Arrow\ X\ (\EndoR\ Y)}}
  (\bind\ f) \rrr (\bind\ g) & ≡ \bind\ (f \fish g)
\end{split}
\end{align}
%
Here likewise the arrows $f \tp \Arrow\ X\ (\EndoR\ Y)$ and $g \tp
\Arrow\ Y\ (\EndoR\ Z)$ are universally quantified (as well as the objects they
range over). $\fish$ is the Kleisli-arrow which is defined as $f \fish g \defeq
f \rrr (\bind\ g)$ . (\TODO{Better way to typeset $\fish$?})

\subsection{Equivalence of formulations}
%
In my implementation I proceed to show how the one formulation gives rise to
the other and vice-versa. For the present purpose I will briefly sketch some
parts of this construction:

The notation I have chosen here in the report
overloads e.g. $\pure$ to both refer to a natural transformation and an arrow.
This is of course not a coincidence as the arrow in the Kleisli formulation
shall correspond exactly to the map on arrows from the natural transformation
called $\pure$.

In the monoidal formulation we can define $\bind$:
%
\begin{align}
\bind \defeq \join \lll \fmap\ f
\end{align}
%
And likewise in the Kleisli formulation we can define $\join$:
%
\begin{align}
\join \defeq \bind\ \identity
\end{align}
%
It now remains to show that we can prove the various laws given this choice. I
refer the reader to my implementation for the details.
