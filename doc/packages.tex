\usepackage[utf8]{inputenc}

\usepackage{natbib}
\usepackage[
  hidelinks,
  pdfusetitle,
  pdfsubject={category theory},
  pdfkeywords={type theory, homotopy theory, category theory, agda}]
  {hyperref}

\usepackage{graphicx}

\usepackage{parskip}
\usepackage{multicol}
\usepackage{amssymb,amsmath,amsthm,stmaryrd,mathrsfs,wasysym}
\usepackage[toc,page]{appendix}
\usepackage{xspace}
\usepackage[a4paper]{geometry}

% \setlength{\parskip}{10pt}

% \usepackage{tikz}
% \usetikzlibrary{arrows, decorations.markings}

% \usepackage{chngcntr}
% \counterwithout{figure}{section}
\numberwithin{equation}{section}

\usepackage{listings}
\usepackage{fancyvrb}

\usepackage{mathpazo}
\usepackage[scaled=0.95]{helvet}
\usepackage{courier}
\linespread{1.05} % Palatino looks better with this

\usepackage{lmodern}

\usepackage{enumerate}
\usepackage{verbatim}

\usepackage{fontspec}
\usepackage[light]{sourcecodepro}
%% \setmonofont{Latin Modern Mono}
%% \setmonofont[Mapping=tex-text]{FreeMono.otf}
%% \setmonofont{FreeMono.otf}


%% \pagestyle{fancyplain}
\setlength{\headheight}{15pt}
\renewcommand{\chaptermark}[1]{\markboth{\textsc{Chapter \thechapter. #1}}{}}
\renewcommand{\sectionmark}[1]{\markright{\textsc{\thesection\ #1}}}

% Allows for the use of unicode-letters:
\usepackage{unicode-math}

%% \RequirePackage{kvoptions}

\usepackage{pgffor}
\lstset
  {basicstyle=\ttfamily
  ,columns=fullflexible
  ,breaklines=true
  ,inputencoding=utf8
  ,extendedchars=true
  %% ,literate={á}{{\'a}}1 {ã}{{\~a}}1 {é}{{\'e}}1
  }
  
\usepackage{newunicodechar}

%% \setmainfont{PT Serif}
\newfontfamily{\fallbackfont}{FreeMono.otf}[Scale=MatchLowercase]
%% \setmonofont[Mapping=tex-text]{FreeMono.otf}
\DeclareTextFontCommand{\textfallback}{\fallbackfont}
\newunicodechar{∨}{\textfallback{∨}}
\newunicodechar{∧}{\textfallback{∧}}
\newunicodechar{⊔}{\textfallback{⊔}}
\newunicodechar{≊}{\textfallback{≊}}
\newunicodechar{∈}{\textfallback{∈}}
\newunicodechar{ℂ}{\textfallback{ℂ}}
\newunicodechar{∘}{\textfallback{∘}}
\newunicodechar{⟨}{\textfallback{⟨}}
\newunicodechar{⟩}{\textfallback{⟩}}
\newunicodechar{∎}{\textfallback{∎}}
\newunicodechar{𝒜}{\textfallback{?}}
\newunicodechar{ℬ}{\textfallback{?}}
%% \newunicodechar{≊}{\textfallback{≊}}
